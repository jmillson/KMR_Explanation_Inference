\documentclass{article}                     % onecolumn (standard format)
%\documentclass[smallcondensed]{svjour3}     % onecolumn (ditto)
%\documentclass[smallextended]{svjour3}       % onecolumn (second format)
%\documentclass[twocolumn]{svjour3}          % twocolumn
%

\usepackage{geometry}
\usepackage{graphicx}
\usepackage{amsmath}
\usepackage{mathptmx}
\usepackage{stmaryrd}
\usepackage{enumitem}
\usepackage{times}
\usepackage{graphicx}
\usepackage{latexsym}
\usepackage{bussproofs}
\usepackage{pgf}
\usepackage{adjustbox}
\usepackage{multirow}
\usepackage{array}
\usepackage{tikz}
\usepackage{xcolor}
\usepackage{ushort}
\usepackage{soul}
\usepackage[autostyle]{csquotes}
\usepackage[doi=false,isbn=false,url=false,style=chicago-authordate,natbib=true]{biblatex}

%\usepackage{fonttable}

\addbibresource{KMR_Master.bib}



\DeclareMathSymbol{\Gamma}{\mathalpha}{operators}{0}
\DeclareMathSymbol{\Delta}{\mathalpha}{operators}{1}
\DeclareMathSymbol{\Theta}{\mathalpha}{operators}{2}
\DeclareMathSymbol{\Lambda}{\mathalpha}{operators}{3}
\DeclareMathSymbol{\Xi}{\mathalpha}{operators}{4}
\DeclareMathSymbol{\Pi}{\mathalpha}{operators}{5}
\DeclareMathSymbol{\Sigma}{\mathalpha}{operators}{6}
\DeclareMathSymbol{\Upsilon}{\mathalpha}{operators}{7}
\DeclareMathSymbol{\Phi}{\mathalpha}{operators}{8}
\DeclareMathSymbol{\Psi}{\mathalpha}{operators}{9}
\DeclareMathSymbol{\Omega}{\mathalpha}{operators}{10}


\DeclareFontFamily{U} {MnSymbolA}{}

\DeclareFontShape{U}{MnSymbolA}{m}{n}{
  <-6> MnSymbolA5
  <6-7> MnSymbolA6
  <7-8> MnSymbolA7
  <8-9> MnSymbolA8
  <9-10> MnSymbolA9
  <10-12> MnSymbolA10
  <12-> MnSymbolA12}{}
\DeclareFontShape{U}{MnSymbolA}{b}{n}{
  <-6> MnSymbolA-Bold5
  <6-7> MnSymbolA-Bold6
  <7-8> MnSymbolA-Bold7
  <8-9> MnSymbolA-Bold8
  <9-10> MnSymbolA-Bold9
  <10-12> MnSymbolA-Bold10
  <12-> MnSymbolA-Bold12}{}

\DeclareSymbolFont{MnSyA}{U}{MnSymbolA}{m}{n}
\DeclareMathSymbol{\twoheaduparrow}{\mathop}{MnSyA}{25}
\DeclareMathSymbol{\twoheadrightarrow}{\mathop}{MnSyA}{24}

\makeatletter


\usetikzlibrary{fit,shapes.misc}

\newcommand\marktopleft[1]{%
	\tikz[overlay,remember picture] 
	\node (marker-#1-a) at (3cm, 1mm) {};%
}
\newcommand\markbottomright[1]{%
	\tikz[overlay,remember picture] 
	\node (marker-#1-b) at (-3cm,7mm) {};%
	\tikz[overlay,remember picture,thick,inner sep=3pt]
	\node[draw,rounded rectangle,fit=(marker-#1-a.center) (marker-#1-b.center)] {};%
}





\newcommand{\ee}{\twoheadrightarrow}

% % % % % % % % % % % % % % % % Footnote Command % % % % % % % % % % % % %
\usepackage{refcount}% http://ctan.org/pkg/refcount
\newcounter{fncntr}
\newcommand{\fnmark}[1]{\refstepcounter{fncntr}\label{#1}\footnotemark[\getrefnumber{#1}]}
\newcommand{\fntext}[2]{\footnotetext[\getrefnumber{#1}]{#2}}

% % % % % % % % % % % % % % % Internal Commands NMC% % % % % % % % % % % % %
\newcommand{\raisemath}[1]{\mathpalette{\raisem@th{#1}}}
\newcommand{\raisem@th}[3]{\raisebox{#1}{$#2#3$}}

\newcommand{\uuparrow}{% 
	\raisebox{.165ex}{\clipbox{0pt .6pt 0pt 0pt}{$\uparrow$}}
}
\newcommand{\tuuparrow}{% 
	\raisebox{.165ex}{\clipbox{0pt 1pt 0pt 0pt}{$\scriptscriptstyle\uparrow$}}
}
\newcommand{\muparrow}{% 
	\raisebox{.05ex}{\clipbox{0pt .65pt 0pt 0pt}{$\scriptstyle\uparrow$}}
}
\newcommand{\Uuparrow}{% 
	\raisebox{.2ex}{\clipbox{0pt .15pt 0pt 0pt}{$\Uparrow$}}
}
\newcommand{\thuarrow}{% 
	\raisebox{.05ex}{\clipbox{0pt .8pt 0pt 0pt}{$\twoheaduparrow$}}
}

% % % % % COMMANDS FOR NON-MONOTONIC CONSEQUENCES % % % % % % % % %
\newcommand{\nms}{%
	\mathbin{\mathpalette\@nms\expandafter}
}
\newcommand{\@nms}{\mid\joinrel\mkern-.5mu\sim}


\newcommand{\nmc}{%
	\mathbin{\mathpalette\nm@\expandafter}
}
\newcommand{\nm@}{\mid\joinrel\mkern-.5mu\sim\mkern-3mu}

\newcommand{\qmc}[1]{\mathrel{
		\mathchoice
		{\normalsize\hspace{.4mm}\nms^{\mkern-18mu\scriptsize\uuparrow#1}\hspace{-.7mm}}
		{\normalsize\hspace{.4mm}\nms^{\mkern-18mu\scriptsize\uuparrow#1}\hspace{-.7mm}}
		{\footnotesize\hspace{.4mm}\nms^{\mkern-13mu\tiny\uuparrow#1}}
		{\scriptsize\nms^{\mkern-10mu\tiny\tuuparrow#1}}
	}
}

\newcommand{\mqmc}{\mathrel{
		\mathchoice
		{\hspace{.4mm}\nms^{\mkern-18mu\scriptsize\uuparrow}\hspace{.6mm}}
		{\hspace{.4mm}\nms^{\mkern-18mu\scriptsize\uuparrow}\hspace{.6mm}}
		{\footnotesize\hspace{.4mm}\nms^{\mkern-11mu\tiny\uuparrow}\hspace{.6mm}}
		{\scriptsize\nms^{\mkern-10mu\tiny\tuuparrow}}
	}
}

\newcommand{\mrc}[1]{\mathbin{
		\mathchoice
		{\normalsize\hspace{.5mm}\nms^{\mkern-19mu\scriptsize\Uuparrow#1}\hspace{-.5mm}}
		{\normalsize\hspace{.5mm}\nms^{\mkern-19mu\scriptsize\Uuparrow#1}\hspace{-.5mm}}
		{\footnotesize\hspace{.5mm}\nms^{\mkern-13.5mu\fontsize{5.5}{0}\Uuparrow#1}}
		{\scriptsize\nms^{\mkern-10mu\tiny\Uuparrow#1}}
	}
}

\newcommand{\smc}{\mathbin{
		\mathchoice
		{\hspace{.4mm}\nms^{\mkern-17mu\scriptsize\thuarrow}\hspace{.6mm}}
		{\hspace{.4mm}\nms^{\mkern-17mu\scriptsize\thuarrow}\hspace{.6mm}}
		{\footnotesize\hspace{.4mm}\nms^{\mkern-11mu\tiny\thuarrow}\hspace{.6mm}}
		{\scriptsize\nms^{\mkern-10mu\tiny\thuarrow}}
	}
}
\newcommand{\nnmc}{\not\nmc}
\newcommand{\nsmc}{\not\mkern-3mu\smc}
\newcommand{\nmrc}{\not\mkern-3mu\mrc}
\newcommand{\nmqmc}{\not\mkern1mu\mqmc}
\newcommand{\nqmc}{\not\mkern1mu\qmc}

\newcommand{\nme}{%
	\mathbin{\mathpalette\@nme\expandafter}
}
\newcommand{\@nme}{{\mid\joinrel\mkern-.5mu\sim\mkern-2mu}_{e}\mkern3mu}


%\newcommand{\nme}{\nms\mkern-7mu_{e}\mkern2mu}
\newcommand{\qme}{{\qmc\mkern-2mu}_{e}\mkern3mu}
\newcommand{\mqme}{{\mqme\mkern-2mu}_{e}\mkern3mu}
\newcommand{\sme}{{mkern-2mu}_{e}\mkern3mu}

% % % % % % % % % % %Commands for Material Incoherence% % % % % % % % % % % %

\newcommand{\bigperpp}{%
	\mathop{\mathpalette\bigp@rpp\relax}%
	\displaylimits
}
\newcommand{\bigp@rpp}[2]{%
	\vcenter{
		\m@th\hbox{\scalebox{\ifx#1\displaystyle1.3\else1.3\fi}{$#1\perp$}}
	}%
}
\newcommand{\bigperp}{\raisemath{.5pt}{\bigperpp}}

%% % % % % % Degree Command % % % % % % % % % % % % % % % % % % % %
\newcommand{\degree}{\ensuremath{^\circ}}

%%%%%%%%%%%%%Author Comments%%%%%%%%%%%%%%%%%
 \newcommand{\kk}[1]{\textcolor{red}{$^{\textrm{KK}}${#1}}}
 \newcommand{\jm}[1]{\textcolor{blue}{$^{\textrm{JM}}${#1}}}
 \newcommand{\mr}[1]{\textcolor{green}{$^{\textrm{MR}}${#1}}}


\newcolumntype{L}[1]{>{\raggedright\let\newline\\\arraybackslash\hspace{0pt}}m{#1}}
\newcolumntype{C}[1]{>{\centering\let\newline\\\arraybackslash\hspace{0pt}}m{#1}}
\newcolumntype{R}[1]{>{\raggedleft\let\newline\\\arraybackslash\hspace{0pt}}m{#1}}



\makeatother





%\renewcommand{\@cite}[1]{#1}
%\usepackage[natbibapa]{apacite}
\usepackage[hang,flushmargin]{footmisc} 
\usepackage[hidelinks]{hyperref}
%\usepackage{lingmacros}
%\hypersetup{
%    colorlinks=false,
%    pdfborder={0 0 0},
%}


\begin{document}
\sloppy
\title{Pragmatics of Explanation}

\raggedbottom

\maketitle


\noindent\textbf{Sturdy  Consequence (SC): [Sets]}\label{SCset}
\begin{equation}
\Gamma, \, \Sigma \smc B \Longrightarrow
\begin{cases}\nonumber 
1.\,\, \Gamma, \, \Sigma \mrc{W} B & $ where $ B \not\in \Sigma $, and $ \\[3pt] 
2.\,\, \forall \Theta\,(\Gamma, \Theta \mrc{W'} B \Longrightarrow W\not\subset W' ) & $ where $ B\not\in \Theta  \\[3pt] 
\end{cases}
\end{equation}


\section{Desiderata for a Pragmatics of Explanation6:}
\begin{itemize}
\item Account for the difference between logically possible answers to a why-question and the contextually possible answers  to a why-question.
\item Treat the propriety of contextually possible answers as a function of the cognitive and conative states/attitudes (or their deontic scorekeeping analogs) of conversational participants.
\item If the alternatives that comprise the semantic content of why-questions are complete explanations, then the pragmatics needs to provide a notion of a \textit{partial answer to a why-Question} as well as the rules that govern which partial answers are the `right ones' given the context (see next point).
\item Define the relationship between complete/partial explanations and complete/partial answers to why-questions.
\item Identify the contextual parameters that determine what makes an explanation (i.e.possible [partial] answer to a why-question) the `best'.
\item Make our picture of explanatory practices more descriptively adequate.
\item Less urgent:
\begin{itemize}
\item Account for the impermissibility of disjunctive explanations.
\item Account for distinction between A and Gamma??
\end{itemize}
\end{itemize}


\section{Initial Thoughts:}
\begin{itemize}
\item \textsc{Hypothesis:} Sturdy inferences ($\Gamma, \Sigma \smc B $) are made explicit by claims of non-factive, complete, immediate explanation ($\Gamma, \Sigma \ee B $).
\begin{itemize}
\item \textit{Non-factive:} A sturdy inference entails neither the explanans nor the explanadum.
\item \textit{Complete:} Nothing need be added to $\Sigma$ in the context of $\Gamma$ in order to explain $B$.
\item \textit{Immediate:} It is not the case that $\Sigma$ explains $B$ in $\Gamma$ by explaining some $C$ which then explains $B$. 
\end{itemize}

\item The set of alternatives which constitutes or in part constitutes the semantic content of why-questions, e.g. \textit{Why B?} are NOT the premises of sturdy inferences (i.e. $\Sigma$), but are the inferences, or the explicit inferential commitments themselves (i.e. $\Gamma, \Sigma \ee B $) with the same conclusion.
\begin{itemize}
\item[***] \textbf{Are the alternatives (1) sturdy inferences (i.e. $\Gamma, \Sigma \smc B \,\,$; $\,\,\Gamma \Theta \smc B $ ; etc.) or (2) merely modally robust inferences (i.e. $\Gamma, \Sigma \mrc{W} B \,\,$; $\,\,\Gamma, \Theta \mrc{W'} B $ ; etc.)?}
\begin{itemize}
\item Option (1) would be in keeping with the standard semantics for questions, but (2) would align with our SC2...
\item Let's go with option (1) and see how far we can run with it.\\
\end{itemize}
\end{itemize}
\end{itemize}

\section{Intro to Partition Semantics and Pragmatics of Answers}

In their dissertation, Groenendijk and Stokhof, provide a partition semantics for questions and a pragmatics for answers. The basic idea is that the semantic content of questions is a partition on logical space that divides the latter into mutually exclusive, exhaustive possibilities. For instance, the question ``Who's in Chicago'' ($ ?xCx $) asked in universe of discourse consisting of only two individuals, \textit{a} and \textit{b}, would divide logical space as follows:

\renewcommand{\arraystretch}{4}
\begin{tabular}{ r|C{4cm}|C{4cm}| }
	\multicolumn{1}{r}{}
	&  \multicolumn{1}{c}{$Ca$}
	& \multicolumn{1}{c}{$\neg Ca$} \\
	\cline{2-3}
	$Cb$ & $Ca \wedge Cb$ & $\neg Ca \wedge Cb$ \\
	\cline{2-3}
	$\neg Cb$ & $Ca \wedge \neg Cb$ & $\neg Ca \wedge \neg Cb$ \\
	\cline{2-3}
\end{tabular}\\

A complete answer to a question is a piece of information that identifies one cell as actual, while a partial answer is any piece of information that permits us to eliminate at least one cell. 

G \& S's pragmatics aims to represent the role that context plays in circumscribing the set of possible answers. If the context is a set of propositions, then the pragmatically relevant possible answers as those that partition the portion of logical space carved out by the context. Consider the following picture, the same as the one for $ ?xCx $, but now with an additional oval which represents the current state of information, i.e. the context:

\renewcommand{\arraystretch}{4}
\begin{tabular}{ r|L{4cm}|R{4cm}| }
	\multicolumn{1}{r}{}
	&  \multicolumn{1}{c}{$Ca$}
	& \multicolumn{1}{c}{$\neg Ca$} \\
	\cline{2-3}
	$Cb$ &	\marktopleft{c1}\raisebox{.7cm}{$Ca \wedge Cb$}
	& \raisebox{.7cm}{$\neg Ca \wedge Cb$} \\
	\cline{2-3}
	$\neg Cb$ & $Ca \wedge \neg Cb$ & $\neg Ca \wedge \neg Cb$\markbottomright{c1} \\
	\cline{2-3}
\end{tabular}\\

\vspace{3mm}

The oval represents the idea that participants assume that actual world lies inside of it, and that all possibilities outside the oval have been dismissed as being non-actual. (Of course, it may be that they have been dismissed mistakenly.) The above picture indicates that, while the semantic question cuts up logical space into four big blocks, it is the division of the oval into four parts that is pragmatically relevant. 

Of course, context can also exclude certain logically possible answers, i.e. exclude cells of the partition of logical space. The next picture represents a context that excludes certain alternatives.

\renewcommand{\arraystretch}{4}
\begin{tabular}{ r|L{4cm}|R{4cm}| }
	\multicolumn{1}{r}{}
	&  \multicolumn{1}{c}{$Ca$}
	& \multicolumn{1}{c}{$\neg Ca$} \\
	\cline{2-3}
	$Cb$ &	\marktopleft{c1}\raisebox{.7cm}{$Ca \wedge Cb$}
	& \raisebox{.7cm}{$\neg Ca \wedge Cb$}\markbottomright{c1}  \\
	\cline{2-3}
	$\neg Cb$ & $Ca \wedge \neg Cb$ & $\neg Ca \wedge \neg Cb$ \\
	\cline{2-3}
\end{tabular}\\

\vspace{2mm}
In this scenario, only two of the logically possible answers are permissible in the context.

\section{Applying Partition Semantics to Why-questions:}

The first step in articulating a partition semantics for why-questions is to determine how why-questions partition logical space. Our guiding insight is that why-questions divide logical space into possible complete explanations.  However, while we have been representing complete explanations as sturdy inferences relativized to contexts (i.e. $\Gamma$), we should omit these contexts in our formalization of the partition induced by why-questions since the semantics already has a mechanism for representing contexts. Thus, the set of alternative associated with the question \textit{Why B?} is $ \{ \Sigma \ee B \,\,,\,\, \Theta \ee B \} $ rather than $\{ \Gamma, \Sigma \ee B \,\,$; $\,\,\Gamma, \Theta \ee B \} $. Following the partition semantics approach, the semantic content of this question is the partition induced by this set, i.e. the set of mutually-exclusive, exhaustive possibilities: $$ \{ \Sigma \ee B  \,\,\&\,\, \Theta \ee B,\,\, \Sigma \ee B \,\,\&\,\,\neg(\Theta \ee B),\,\, \neg(\Sigma \ee B) \,\,\&\,\,\Theta \ee B,\,\, \neg(\Sigma \ee B)\,\,\&\,\,\neg(\Theta \ee B)       \} $$  

The partition looks like this:\\

\renewcommand{\arraystretch}{4}
\begin{tabular}{ r|C{4cm}|C{4cm}| }
	\multicolumn{1}{r}{}
	&  \multicolumn{1}{c}{$\Sigma \ee B$}
	& \multicolumn{1}{c}{$\neg(\Sigma \ee B)$} \\
	\cline{2-3}
	$\Theta \ee B$ & $\Sigma \ee B \,\,$ \& $\,\,\Theta \ee B$ & $\neg(\Sigma \ee B) \,\,\&\,\, \Theta \ee B$ \\
	\cline{2-3}
	$\neg(\Theta \ee B)$ & $\Sigma \ee B \,\,$ \& $\,\,\neg(\Theta \ee B)$ & $\neg(\Sigma \ee B)\,\,\&\,\,\neg(\Theta \ee B)$ \\
	\cline{2-3}
\end{tabular}\\




%\renewcommand{\arraystretch}{3}
%\begin{tabular}{|c|c|}
%\hline $ \Sigma \ee B \,\,$ \& $\,\,\Theta \ee B $
% & $ \Sigma \ee B \,\,$ \& $\,\,\neg(\Theta \ee B) $ \\ 
%\hline $ \neg(\Sigma \ee B) \,\,\&\,\, \Theta \ee B $
% & $ \neg(\Sigma \ee B)\,\,\&\,\,\neg(\Theta \ee B) $ \\ 
%\hline 
%\end{tabular} \\\\

\vspace{3mm}

Now here is the same question, \textit{Why B?} asked in context $\Gamma$. Here, the oval represents the set of commitments $\Gamma$.

\renewcommand{\arraystretch}{4}
\begin{tabular}{ r|C{4cm}|C{4cm}| }
	\multicolumn{1}{r}{}
	&  \multicolumn{1}{c}{$\Sigma \ee B$}
	& \multicolumn{1}{c}{$\neg(\Sigma \ee B)$} \\
	\cline{2-3}
	$\Theta \ee B$ & \marktopleft{c1}\raisebox{.7cm}{$\Sigma \ee B \,\,$ \& $\,\,\Theta \ee B$} & \raisebox{.7cm}{$\neg(\Sigma \ee B) \,\,\&\,\, \Theta \ee B$} \\
	\cline{2-3}
	$\neg(\Theta \ee B)$ & $\Sigma \ee B \,\,$ \& $\,\,\neg(\Theta \ee B)$ & $\neg(\Sigma \ee B)\,\,\&\,\,\neg(\Theta \ee B)$\markbottomright{c1} \\
	\cline{2-3}
\end{tabular}\\

\vspace{5mm}

Just as in the case of \textit{Who}-question, why-questions can be asked in contexts in which some logically possible answers are not contextually permissible. In the case of why-questions, this means that certain possible complete explanations are excluded from consideration.

\renewcommand{\arraystretch}{4}
\begin{tabular}{ r|C{4cm}|C{4cm}| }
	\multicolumn{1}{r}{}
	&  \multicolumn{1}{c}{$\Sigma \ee B$}
	& \multicolumn{1}{c}{$\neg(\Sigma \ee B)$} \\
	\cline{2-3}
	$\Theta \ee B$ & \raisebox{.7cm}{$\Sigma \ee B \,\,$ \& $\,\,\Theta \ee B$}\marktopleft{c1} & \raisebox{.7cm}{$\neg(\Sigma \ee B) \,\,\&\,\, \Theta \ee B$} \\
	\cline{2-3}
	$\neg(\Theta \ee B)$ & $\Sigma \ee B \,\,$ \& $\,\,\neg(\Theta \ee B)$ & $\neg(\Sigma \ee B)\,\,\&\,\,\neg(\Theta \ee B)$\markbottomright{c1} \\
	\cline{2-3}
\end{tabular}\\

\vspace{5mm}

In this situation, the context, $\Sigma$ is not a permissible answer to the question \textit{Why B?}, and hence is not a competitor for the status of complete explanation of $B$.

\section{Pragmatics of Answers and Explanation}

Recall that in partition semantics for questions, a partial answer is any piece of information that permits us to eliminate at least one cell. So, any of the following would be a partial answer to the question ``Why B?'' asked in $\Gamma$: e.g. $\Sigma \ee B\,,\,\, \neg(\Theta \ee B)\,,\,\, \neg(\Sigma \ee B)\,,\,\, \Theta \ee B  $. 

But in most conversational contexts, we answer why questions by giving only a partial explanation. In fact, complete explanations seem to be very rarely called for. Given, the way we have defined \textit{partial answer to a why-question}, we can say that \textbf{a partial explanation is a part of a partial answer to a why-question.} For example, to the question ``Why B?'' a normal response would be ``Because A'' where $ A \in \Sigma $. In a context that includes all the cells of the partition above, this response will conversationally entail the partial answer $\Sigma \ee B  $.\\

\textbf{Definition:} A partial explanation of B is a part of a partial answer to the question ``Why B?''\\

\begin{itemize}
	\item A big difference between our pragmatics and that of G\&S is that we conceive of context as a set of commitments (and entitlements?) that consists of both the doxastic and the practical variety. (Perhaps also erotetic and apokritic).
	\item We can successfully do so because we are working on the basis of material inferences and there are certainly practical material inferences.
\end{itemize}

******Gamma weeds out alternative explanations by restricting the sets of suppositions that may be added to the material inferences. This follows from MR3*****












%\noindent\textbf{Context-Set}\\
%\noindent $  \Pi \in \mathcal{L}   $\\


%\noindent\textbf{Sturdy  Consequence (SC):}\label{SC}
%\begin{equation}
%	\Gamma, \,A \smc B \Longrightarrow
%	\begin{cases}\nonumber 
%		1.\,\, \Gamma, \, A \mrc{W} B & $ where $ A \neq B $, and $ \\[3pt] 
%		2.\,\, \forall C\,(\Gamma, C \mrc{W'} B \Longrightarrow W\not\subset W' ) & $ where $ B\neq C  \\[3pt] 
%	\end{cases}
%\end{equation}
%
%\noindent\textbf{Sturdy  Consequence (SC): [Sets]}\label{SCset}
%\begin{equation}
%\Gamma, \, \Sigma \smc B \Longrightarrow
%\begin{cases}\nonumber 
%1.\,\, \Gamma, \, \Sigma \mrc{W} B & $ where $ B \not\in \Sigma $, and $ \\[3pt] 
%2.\,\, \forall \Theta\,(\Gamma, \Theta \mrc{W'} B \Longrightarrow W\not\subset W' ) & $ where $ B\not\in \Theta  \\[3pt] 
%\end{cases}
%\end{equation}



\printbibliography


\end{document}
