\documentclass{article}
\usepackage[top=1in,bottom=1in,left=1.75in,right=1in]{geometry}
\usepackage{setspace}
\usepackage{amsmath}
\usepackage{amsthm}
\usepackage{amssymb}
\usepackage{stmaryrd}
\usepackage{natbib}
\usepackage{enumitem}
\usepackage{times}
\usepackage{graphicx}
\usepackage{latexsym}
\usepackage{bussproofs}
\usepackage{MnSymbol}
\usepackage{wasysym}
\usepackage{turnstile}
\usepackage{pgf}
\usepackage{adjustbox}% http://ctan.org/pkg/adjustbox
\usepackage{xcolor}
\usepackage{soul}
\renewcommand{\makehor}[4]
  {\ifthenelse{\equal{#1}{n}}{\hspace{#3}}{}
   \ifthenelse{\equal{#1}{s}}{\rule[-0.5#2]{#3}{#2}}{}
   \ifthenelse{\equal{#1}{d}}{\setlength{\lengthvar}{#2}
     \addtolength{\lengthvar}{0.5#4}
     \rule[-\lengthvar]{#3}{#2}
     \hspace{-#3}
     \rule[0.5#4]{#3}{#2}}{}
   \ifthenelse{\equal{#1}{t}}{\setlength{\lengthvar}{1.5#2}
     \addtolength{\lengthvar}{#4}
     \rule[-\lengthvar]{#3}{#2}
     \hspace{-#3}
     \rule[-0.5#2]{#3}{#2}
     \hspace{-#3}
     \setlength{\lengthvar}{0.5#2}
     \addtolength{\lengthvar}{#4}
     \rule[\lengthvar]{#3}{#2}}{}
   \ifthenelse{\equal{#1}{w}}{% New wavy $\sim$ definition
     \setbox0=\hbox{$\sim$}%
     \raisebox{-.6ex}{\hspace*{-.05ex}\adjustbox{width=#3,height=\height}{\clipbox{0.75 0 0 0}{\usebox0}}}}{}
  }

\makeatletter
\newcommand{\bigperp}{%
  \mathop{\mathpalette\bigp@rp\relax}%
  \displaylimits
}
\newcommand{\bigp@rp}[2]{%
  \vcenter{
    \m@th\hbox{\scalebox{\ifx#1\displaystyle2.1\else1.5\fi}{$#1\perp$}}
  }%
}
\makeatother

\DeclareSymbolFont{symbolsC}{U}{ntxsyc}{m}{n}
\SetSymbolFont{symbolsC}{bold}{U}{ntxsyc}{n}{b}
\DeclareMathSymbol{\multimapdotbothA}{\mathrel}{symbolsC}{23}
\DeclareMathSymbol{\boxright}{\mathrel}{symbolsC}{128}

\usepackage[hidelinks]{hyperref}
%\usepackage{lingmacros}
\hypersetup{
    colorlinks=false,
    pdfborder={0 0 0},
}
\newcommand\pref[1]{(\ref{#1})}
%\usepackage{calc}
%\usepackage{covington}
%\usepackage{diagbox}
%\usepackage{fixltx2e}
%\usepackage{tikz}
%\usetikzlibrary{calc}

%\newcommand{\hcancel}[5]{%
%    \tikz[baseline=(tocancel.base)]{
%        \node[inner sep=0pt,outer sep=0pt] (tocancel) {#1};
%        \draw[red] ($(tocancel.north west)$) -- ($(tocancel.south east)$);}}%

%\DeclareFontEncoding{LGR}{}{}
%\DeclareTextSymbol{\~}{LGR}{126}

%\providecommand{\tabularnewline}{\\}
%\newcommand*{\equationautorefname}[1]{\@gobble}
\setcitestyle{aysep={}, notesep=:}

%opening
\title{Nonmonotonic material inferences and explanations}
\author{Jared Millson}
\date{}
\raggedbottom


\newcommand{\nc}{\turnstile{s}{w}{}{}{n}}
%\newcommand{\nc}{\,\mid\!\sim\,}

\begin{document}
\maketitle
\setlength{\parindent}{1cm}
\large

\textbf{{\Large Thinking through some things...}}\\

We start with a material consequence relation and an incoherence property over a language $ \mathcal{L} $ devoid of logical vocabulary. Let $ p, p_1, p_2, $ etc. stand for atomic sentences, $ \Gamma, \Delta $ stand for sets of atomic sentences, X, Y, Z for sets of sets of atomic sentences, and $ \bigperp $ denote the incoherence property. 

Let $ \nc $ and $ \bigperp $ be defined over $ \mathcal{L} $ such that $  \Gamma\nc s $ iff  $\Gamma$ materially implies $ s $, and $\Gamma\nc \bigperp $ iff $ \Gamma $ is incoherent. 

The structure  $ \langle \mathcal{L}, \nc, \bigperp \rangle  $ has the following properties: 
\begin{enumerate}
\item $  \mathcal{L}\nc\bigperp$
\item $ \varnothing\not\nc\bigperp $
\item $\forall \Gamma\in\mathcal{L} (p \in \Gamma \Rightarrow \Gamma\nc p)$ ( Reflexivity)
\item  If $ \forall\Delta\subseteq\mathcal{L}(\Gamma, \Delta, \nc\bigperp),$  then  $\Gamma\nc p $ (Ex Falso Fixo Quodlibet)
\item $\Gamma\nc p \not\Rightarrow \Gamma, \Delta \nc A$ (Non-monotonicity)
\end{enumerate}


\noindent\textbf{Quantified Consequence Relations}
\begin{itemize}
\item [\textbf{Assumption 1}] We assume throughout this paper that $ \Gamma \neq \emptyset $
\item[\textbf{Definition 1}] $\Gamma\nc^{\uparrow X} p \Leftrightarrow_{def}  X\subseteq\mathcal{P}(\mathcal{L}) $ and $ \forall\Delta\in X(\Gamma, \Delta \nc p )$

%\item[\textbf{Definition 2}] $\Gamma\nc^{\downarrow \Sigma}p \Leftrightarrow_{def}  \Sigma \subset \Gamma $ and $ \forall\Delta\subseteq \Sigma(\Gamma - \Delta \nc p )$

%\item[\textbf{Definition 3}] $ \Gamma\nc^{\uparrow X \cup\downarrow \Sigma} p \Leftrightarrow_{def} X \subseteq\mathcal{P}(\mathcal{L})$ and $ \Sigma \subset \Gamma $ and  $  \forall\Delta\in X(\Gamma, \Delta \nc p ) $ and $ \forall\Delta\subseteq \Sigma(\Gamma - \Delta \nc p ) $

%\item Note that these definitions give quantified consequence relations the following property:  $ \Gamma\nc^{\uparrow X} p \not\Rightarrow \Gamma\nc p $ and $ \Gamma\nc^{\downarrow \Sigma} p \not\Rightarrow \Gamma\nc p $. We simply don't know whether $ \emptyset \subset X/\Sigma $. We could of course build in the requirement that $\Gamma\nc p $.

\vspace{1cm}

%\textbf{Strategy:} Define `Best explains' in terms of `Greatest Maximal Premise-Expansion (i.e. define it in the meta-language via  partially-ordered  quantified consequence relations), then treat `$ \rhd $' as an object-language expression for making this explicit.
\item[\textbf{Subjunctive Robustness:}] $\Gamma, A\nc^{\Uparrow X} B \Leftrightarrow_{def}  \Gamma-\{A\}\not\nc B \text{\,\,but\,\,}\Gamma, A \nc^{\uparrow X} B $ and $ \forall Y \subseteq\mathcal{P}(\mathcal{L}) $ if $ \Gamma, A \nc^{\uparrow Y}B$ then $ Y \subseteq X$.

\item $ \Gamma, A\nc^{\Uparrow X} B $ represents the extent to which $ \Gamma, A \nc B $  is subjunctively robust.



 
 \color{black}

%\item[\textbf{Partial order on $ \nc^{\Lsh} $: }] $\Gamma, A \nc^{*\Lsh X} B  \Leftrightarrow_{def} \text{\,\,If\,\,} \Gamma, A\nc^{\Lsh X'} B \text{\,then\,\,} \forall C\in\mathcal{L}(\Gamma, C\nc^{\Lsh X''} B)\,X''\subseteq X' $

\item[\textbf{Superlative SR}] $\Gamma, A \nc^{\twoheaduparrow} B  \Leftrightarrow_{def}  \Gamma, A\nc^{\Uparrow X} B \text{\,and\,\,} \forall C\in\mathcal{L}, $ if  $\Gamma, C\nc^{\Uparrow Y} B$ then $Y\subseteq X $.


%\item[\textbf{Superlative SR*}] $\Gamma, A \nc^{\twoheaduparrow} B  \Leftrightarrow_{def}  \Gamma-\{A\}\not\nc B \text{\,but\,\,}  \Gamma,A\nc^{\Uparrow X} B \text{\,and\,\,} \forall C\in\mathcal{L}, $ if  $\Gamma-\{C\}\not\nc B \text{\,but\,\,}  \Gamma, C\nc^{\Uparrow Y} B$ then $Y\subseteq X $.


%\item $\Gamma, A \nc^{*\Lsh X} B$ says that  A implies B in context $ \Gamma $ with greater premise-expansion than any other sentence that implies B in the context of $ \Gamma $.
%%\item Giving a conditional definition will ensure that Max$^{\Gamma}_B (A)$ does not mean simply that A is the best explanation of B, and thus our elimination rule will not be redundant. Other motivations??
%\item Does $\Gamma, A \nc^{*\Lsh X} B \,\, = $ A best explains B in context $\Gamma$?
%\item There's no mention of virtues. But does  $\Gamma, A \nc^{*\Lsh X} B $ capture every thing that we want the virtues to do? Are the virtues an heuristic for getting  $\Gamma, A \nc^{*\Lsh X} B$?
%\item The underlying thought is kinda wacko: we use degrees of monotonicity to rank explanations. A better explains B relative to the alternatives the closer it comes to deductively entailing B! Is this undesirable?
%\item Crazy thought:


\begin{prooftree}
\def\fCenter{\ \nc\ }
\AxiomC{$\Gamma, A \nc^{\twoheaduparrow} B $}
\RightLabel{\hspace{5mm}  $\rhd$-R}
\UnaryInf$\Gamma \fCenter A \rhd B$
\end{prooftree}


\begin{prooftree}
\def\fCenter{\ \nc\ }
\AxiomC{$\Gamma, A \nc^{\twoheaduparrow} B $}
\AxiomC{$\Gamma \nc B $}
\RightLabel{\hspace{5mm}  $\rhd$-L}
\BinaryInf$\Gamma, B, A \rhd B \fCenter A$
\end{prooftree}

%or
%\begin{prooftree}
%\def\fCenter{\ \nc\ }
%\AxiomC{$\Gamma, A \nc^{\twoheaduparrow} B $}
%\AxiomC{$\Gamma, A \nc D $}
%\RightLabel{\hspace{5mm}  $\rhd$-L}
%\BinaryInf$\Gamma, B, A \rhd B \fCenter D$
%\end{prooftree}

\item Notice the lack of quantified consequence relations in the Left rule!
\item If these work, then I see no reason to include cut. Without cut, $ \rhd $ inherits the intransivity of $ \nc $.
\vspace{3cm}

%\textbf{Other Shit I'm toying with...}
%
%
%\item[\textbf{ $\Gamma$-Expansion}]
%\begin{equation}
%    \Gamma\nc^{\Uparrow X}A \Leftrightarrow_{def} 
%    \begin{cases}\nonumber
%      \forall Y \subseteq\mathcal{P}(\mathcal{L})(\Gamma \nc^{\uparrow Y}A) \,\, X \subseteq Y, & \text{if}\ \Gamma \not\nc A \\
%      X = \emptyset, & \text{otherwise}
%    \end{cases}
%  \end{equation}
%  
%  \item $ \Gamma\nc^{\Uparrow X}A $ says that X is the minimal $ \Gamma $-expansion that implies A.
%  \item Since $ \Gamma, A \nc A $ is not (no longer) an axiom, I don't think I need to stipulate that $ A \not\in X $.
%  
%\item[\textbf{$\Gamma$-Contraction}]
%  \begin{equation}
%      \Gamma\nc^{\Downarrow \Sigma}A \Leftrightarrow_{def} 
%      \begin{cases}\nonumber
%        \forall \Delta \subset \Gamma (\Gamma \nc^{\downarrow \Delta}A) \,\, \Sigma \subseteq \Delta, & \text{if}\ \Gamma \not\nc A \\
%        \Sigma = \emptyset, & \text{otherwise}
%      \end{cases}
%    \end{equation}
%    
%    \item $ \Gamma\nc^{\Downarrow \Sigma}A $ says that $ \Sigma $ is the minimal $ \Gamma $-contraction that implies A.
%    
%    \item Weird Thought:
%    \begin{prooftree}
%    \def\fCenter{\ \nc^{\Uparrow X}\ }
%    \AxiomC{$\Gamma \nc^{\Uparrow X} A $}
%    \AxiomC{$\Gamma \nc^{\Uparrow Y} B$}
%	\AxiomC{$X \subseteq Y$}
%    \TrinaryInf$\Gamma, A \fCenter B  $
%    \end{prooftree}

%since $ \emptyset \subset X: X \subseteq \mathcal{P}(\mathcal{L}),\, \Gamma\nc^{\uparrow X \cup\downarrow \Sigma} A \,\, \Rightarrow \Gamma\nc A $, 
%
%\item If $ \Gamma\nc^{\uparrow X \cup\downarrow \Sigma} A $, then $ \Gamma $ implies A under all the revisions of $ \Gamma $ that result from adding subsets of X or subtracting subsets of \Sigma.
%
%\item By convention we will use W to stand for $ \{X \cup \Sigma\} $ such that $ \Gamma\nc^{\uparrow X \cup\downarrow \Sigma} A = \Gamma\nc^{W} A$.
%\item When we wish to restrict W such that for some set of sentences $ \Delta \subseteq\mathcal{L}, \Delta \not\subseteq W$, we will write $ W - \Delta $, e.g. $ \Gamma\nc^{W-\Delta} A$.
%
%
%\item Here's a quantified (restricted) version of Cut that is available:
%\begin{prooftree}
%\def\fCenter{\ \nc^{W}\ }
%\AxiomC{$\Gamma \nc^{W} A $}
%\AxiomC{$\Gamma, A \nc^{W} B$}
%\RightLabel{\hspace{5mm}  \textbf{Quantified Cut}}
%\BinaryInf$\Gamma \fCenter B  $
%\end{prooftree}
%
%
%
%
%
%
%\noindent\textbf{Definitions 4 and 5  are very speculative}
%\item [\textbf{Definition 4}] $ \Gamma\nc^{Max(W)} A \Leftrightarrow_{def} \neg\exists\Delta\subseteq\mathcal{L}(\Gamma, \Delta \nc A $ and $ \Delta \not\subseteq W) $ and $\neg\exists\Delta\subseteq\mathcal{L}(\Delta \subset \Gamma $ and $\Gamma - \Delta \nc A $ and $ \Delta \not\subseteq W)$
%
%
%\item If $ \Gamma\nc^{Max(W)} A $, then W contains all the sets of sentences that can be added to or removed from $ \Gamma $ such that $ \Gamma $ still implies A.
%
%\item [\textbf{Definition 5}] $ \Gamma\nc^{Min(W)} A $ iff $ W \neq\emptyset $ and $\Gamma\not\nc A $ but $ \Gamma\nc^{W} A $ and formalize: If you add or remove one more set of sentences to/from W, then $ \Gamma\not\nc^{W} A $ .


\end{itemize}
%\vspace{3cm}
%\textbf{Possible Model-theoretic Targets}
%
%\begin{itemize}
%\item $ w' \in \Gamma $ and $ w'' \not\in \Gamma \Rightarrow w'<_\Gamma w''  $
%\item $Min_\Gamma (A) = \{w \in A | \forall w'\in A, w'<_\Gamma w $ is false $ \} $ 
%\item Belief-set $ \Gamma $ revised by A, i.e. $ \Gamma ^* A  = Min_\Gamma (A)$
%\end{itemize}


%\Gamma \nc^{\Theta} A $, then $ \Delta\subseteq\Theta $ iff $ \Gamma, \Delta \nc A $ or $ \Gamma - \Delta \nc A$

%in other words, assuming $\Delta\subseteq\mathcal{P}(\mathcal{L}) $ :
%\begin{enumerate}
%\item If $ \Gamma \nc^{\Theta} A $, then $ \Delta\subseteq\Theta $ iff $ \Gamma, \Delta \nc A $ or $ \Gamma - \Delta \nc A $.
%\item One concern about introducing $ \Theta $ as $ \Gamma$'s a maximally-revisable-implication-retaining set is that it will need to be indexed to particular sequents. That is, if $ \Gamma \nc^{\Theta_1} A $ and  $ \Gamma \nc^{\Theta_2} B $, then $ \Theta_1 $ might not be $ \Theta_2 $. We need a convention to deal with this. Maybe the subscript is sufficient.
%\end{enumerate}

%
%\section{REBOOT}
%
%We help ourselves to Hlobil's rules for $ \nc $ and add/specify the following:
%
%\begin{enumerate}
%\item[NN] We will write  $ \nc^{\Theta} $ for Hlobil's $ \nc^{\uparrow X} $
%\item[\textbf{(DTCC)}]  $ \Gamma, A\nc^{\Theta} B $ iff $ \Gamma\nc^{\Theta} A > B $ [DT for Counterfactual Conditionals]
%\item[\textbf{(CCEE)}] $ \Gamma\nc^{\Theta} A > B \Leftrightarrow  \Gamma\nc^{\Theta} A \rhd B$ [CC-`possibly explains' equivalence]
%\end{enumerate}
%\begin{itemize}
%\item Let $\preceq$ be a partial order on sentences, i.e. an reflexive, transitive relation between sentences.
%
%\item $A \preceq_{B}^{\Gamma, \Theta} C$ = `A  is at least as good an explanation of B as C, given $ \Gamma $ updated by any subset of $\Theta $'
%
%\item $A \preceq_{B}^{\Gamma, \Theta} C \Rightarrow \Gamma \nc^{\Theta} A \rhd B $ and $ \Gamma \nc^{\Theta} C \rhd B $ 
%
%\item $ A\preceq_{B}^{\Gamma, \Theta} \phi $ = `A  is the best explanation of B given $ \Gamma $ updated by any subset of $\Theta $', 
%
%\item $ A\preceq_{B}^{\Gamma, \Theta} \phi \Leftrightarrow  $ 
% $ \phi \in \forall C (C\preceq_B^{\Gamma, \Theta} A \Rightarrow A\preceq_B^{\Gamma, \Theta} C)$.
%
%\item We omit $ \Gamma $ and $ \Theta $ when there is no threat of ambiguity.
%
%\item  Rules for A  $ \RHD $ B, i.e. `A best explains why B'.
%
%\begin{prooftree}
%\def\fCenter{\ \nc^{\Theta}\ }
%\AxiomC{$\Gamma \nc^{\Theta} A \rhd B$}
%\AxiomC{$\Gamma \nc^{\Theta} A\preceq_B \phi $}
%\RightLabel{\hspace{5mm}  $\RHD$-R}
%\BinaryInf$\Gamma \fCenter A  \RHD B$
%\end{prooftree}
%
%\begin{prooftree}
%\def\fCenter{\ \nc^{\Theta}\ }
%\AxiomC{$\Gamma \nc^{\Theta} A  \RHD B $}
%\AxiomC{$\Gamma \nc^{\Theta} B$}
%\RightLabel{\hspace{5mm}  $\RHD$-L*}
%\BinaryInf$\Gamma\fCenter A $
%\end{prooftree}
%*=Not a proper Left-Rule, but get's the IBE form.
%
%\begin{prooftree}
%\def\fCenter{\ \nc^{\Theta}\ }
%\AxiomC{$\Gamma \nc^{\Theta} A  \RHD B $}
%\AxiomC{$\Gamma \nc^{\Theta} B$}
%\RightLabel{\hspace{5mm}  $\RHD$-L(Maybe)}
%\BinaryInf$\Gamma, A  \RHD B, B\fCenter A $
%\end{prooftree}
%
%\begin{prooftree}
%\def\fCenter{\ \nc^{\Theta}\ }
%\AxiomC{$\Gamma, A  \RHD B \nc^{\Theta} B $}
%\AxiomC{$\Gamma, A  \RHD B, A \nc^{\Theta} C$}
%\RightLabel{\hspace{5mm}  $\RHD$-L(Maybe)}
%\BinaryInf$\Gamma, A  \RHD B, B\fCenter C $
%\end{prooftree}
%\end{itemize}
%
%
%\newpage
%
%
%
%Argh...I can't seem to get the elimination rule (modus tollens) without Cut...which means giving up monotonicity or the deduction theorem:
% Using Hlobil's rules CCP and LN, I can only get the right inference by using cut.
% 
% 
% \begin{prooftree}
% \def\fCenter{\ \nc^{\uparrow X}\ }
% \Axiom$\Gamma \fCenter \neg A > \neg B$
% \RightLabel{\hspace{5mm} CCP}
% \UnaryInf$\Gamma, \neg A \fCenter \neg B$
% \RightLabel{\hspace{5mm} LN}
%  \UnaryInf$\Gamma, \neg A, B \fCenter \bigperp$
% \AxiomC{$\Gamma \nc^{\uparrow X} B$} 
% \RightLabel{\hspace{5mm} CUT}
% \BinaryInf$\Gamma, \neg A \fCenter \bigperp$
%  \RightLabel{\hspace{5mm} LN}
% \UnaryInf$\Gamma \fCenter A$
% \end{prooftree}
% 
% 
%I don't see how to get the intro and elimination rules to harmonize without including CUT, which means abandoning the idea that counterfactual conditions are expressive of material inferences (deduction theorem).
%
%Jared: I'm adding this to your document so I get to take advantage of your cool definitions.\\
%
%It seems to me that the following characterizes the sort of invariance that Lewis's similarity relation and Woodward's counterfactuals do:\\
%$ \Gamma\nc^{A\in\Theta} B \; \Leftrightarrow \; \Gamma\nc^\Theta A>B$ \textcolor{red}{Agreed (JM)}
%
%Also, isn't it a property of Hlobil's system that: 
%$ \Gamma, A \nc^{\Theta} B \; \Leftrightarrow \; \Gamma\nc^{A\cup\theta}B$
%
%\textcolor{red}{ $ \Gamma, A \nc^{\Theta} B \; \Rightarrow \; \Gamma\nc^{A\cup\theta}B$ Yes! 
%Not sure about $ \Gamma, A \nc^{\Theta} B \; \Leftarrow \; \Gamma\nc^{A\cup\theta}B$ (JM)}
%
%I think that under these curcumstances, we can prove that: 
%$ \Gamma, \neg A > \neg B, B \nc^{\Theta} A $ 
%
%No time tonight for the proof, but think I've sketched it on my whiteboard.
%



\end{document}