\documentclass{article}
\usepackage[top=1in,bottom=1in,left=1.25in,right=1.25in]{geometry}
\usepackage{setspace}
\usepackage{amsmath}
\usepackage{amsthm}
\usepackage{amssymb}
\usepackage{stmaryrd}
\usepackage{natbib}
\usepackage{enumitem}
\usepackage{times}
\usepackage{graphicx}
\usepackage{latexsym}
\usepackage{bussproofs}
\usepackage{MnSymbol}



\DeclareSymbolFont{symbolsC}{U}{ntxsyc}{m}{n}
\SetSymbolFont{symbolsC}{bold}{U}{ntxsyc}{n}{b}
\DeclareMathSymbol{\multimapdotbothA}{\mathrel}{symbolsC}{23}
\DeclareMathSymbol{\boxright}{\mathrel}{symbolsC}{128}

\usepackage[hidelinks]{hyperref}
%\usepackage{lingmacros}
\hypersetup{
    colorlinks=false,
    pdfborder={0 0 0},
}
\newcommand\pref[1]{(\ref{#1})}
%\usepackage{calc}
%\usepackage{covington}
%\usepackage{diagbox}
%\usepackage{fixltx2e}
%\usepackage{tikz}
%\usetikzlibrary{calc}

%\newcommand{\hcancel}[5]{%
%    \tikz[baseline=(tocancel.base)]{
%        \node[inner sep=0pt,outer sep=0pt] (tocancel) {#1};
%        \draw[red] ($(tocancel.north west)$) -- ($(tocancel.south east)$);}}%

%\DeclareFontEncoding{LGR}{}{}
%\DeclareTextSymbol{\~}{LGR}{126}

%\providecommand{\tabularnewline}{\\}
%\newcommand*{\equationautorefname}[1]{\@gobble}
\setcitestyle{aysep={}, notesep=:}

%opening
\title{Inferentialism for Best Explanations}
\author{Kareem Kalifa, Jared Millson, Mark Risjord}
\date{}
\raggedbottom


%\newtheorem*{pppq}{Basic Proto-Polar Queries}
%\newtheorem*{ppq}{Proto-Polar ISQs}
%\newtheorem*{rsq}{Reason-Seeking Queries}
%\newcounter{acounter}
%\newcounter{qcounter}
%\newcounter{dcounter}
\newcommand{\nc}{\,\mid\!\sim\,}

\begin{document}
\maketitle
\setlength{\parindent}{1cm}

Let $\alpha, \beta, \phi, \psi$ stand for sentences of a first-order language $\mathcal{L}$. The formula $\langle \alpha \rangle_{s}^{X}$ says that a conversational participant, \textit{X}, holds some normative attitude(s), \textit{s}, toward the assertion of  sentence $\alpha$. The variable \textit{s} takes the value \textit{c} when X is committed to assert $\alpha$ and  \textit{e} when X is entitled to assert $\alpha$. We will refer to these formulas as \textit{normative assignments}. Let \textit{A, B, C,} range over normative assignments and $\Gamma, \Delta, \Theta$ range over sets of normative assignments. The set of all normative assignments attributed to X is X's scorecard, $\Gamma^{X}$.  In what follows, we assume all inferences are  undertaken by either a single agent or by different agents on the basis of a common ground of normative assignments, in which case $\Gamma$ can be thought of as representing the conversational common ground or context. In either case, the specification of participants is unnecessary and therefore has been omitted.

Sequent calculus versions of formal proof theory can now be applied to explain the constraints on scorecards. $\Delta$ $\vdash$ $\Theta$ is a structural constraint on scorecards such that a scorecard satisfies it if and only if it is not the case that the it contains every attribution in $\Delta$ and none in $\Theta$. General constraints on scorecards are analogous to the normal structural rules for the consequence relation, i.e., \textit{reflexivity}, \textit{weakening}, and \textit{cut}.

A special case of the constraint is the form $\Gamma\vdash$. \, A scorecard is said to violate this constraint when it contains every assignment in $\Gamma$, and respects it when there is some assignment in $\Gamma$ that it avoids. $\Gamma$ $\vdash$ says that the assignments in $\Gamma$ are jointly incompatible. For clarity, $\Gamma\vdash$ will be written as $\perp\Gamma$ and where $\Gamma$ = \{A, B\}, $\perp\Gamma$ can be written as A $\perp$ B.
%This is the primitive notion of incompatibility that the system uses to build up incompatibility relations among specific normative statuses.

We now use the standard constraints of $\vdash$ to define a nonmonotonic structure over normative assignments. A set of normative assignments nonmonotonically  $\Gamma\nc A$ \textit{iff} $\Gamma\vdash A$ and if $A \perp B$, then $ B \not\in \{C: \Gamma\vdash C\}$. By definition, the nonmonotonic consequence relation $\nc$ satisfies \textit{reflexivity} and \textit{supraclassicality}, but we stipulate that it also satisfies \textit{cautious monotonicity}:

\begin{prooftree}
\def\fCenter{\mbox{\ $\nc$\ }}
\AxiomC{$\Gamma \nc A$}
\AxiomC{$\Gamma \nc B$}
\RightLabel{  \hspace{5mm} Caut. Mono.}
\BinaryInf$\Gamma, A \fCenter B$

\end{prooftree}\

Let `$\medtriangleright$' be a binary operator expressing a thick-bundle relation between sentences. 
%Let $W$ be the set of all normative assignments belonging to participants of a conversation and let $\{\alpha : \Delta \vdash \alpha\}$ represent the closure of  a set of assignments $\Delta$ under $\vdash$. \hspace{.5mm}Our system has a single default theory $\langle W,D\rangle$ where $D$ is a set of defaults composed of instances of the following form with particular action-types substituted for F and particular agents for $\alpha, \beta$.
%
%$$\frac{\textsf{d}\textsc{F}\alpha : \textsf{e}\textsc{F}\alpha}{\textsf{e}\textsc{F}\alpha}$$
%
%%Given a pair of sets of assignments ($X, Y$), a default is trigger by ($X, Y$) \emph{iff} $X\vdash \textsf{d}\textsc{F}\alpha$ and $Cn(Y)\not\perp \textsf{e}\textsc{F}\alpha$. 
%Rules of this form say that if an agent has F-ed and there is no assignment incompatible with its entitlement, then the agent is entitled to F. Extensions of $\langle W,D\rangle$ are defined by fixed point construction. For any set of normative assignments, $X$, let $\Gamma(X)$ be the least closed set that includes $W$ and satisfies the following condition: If $\textsf{d}\textsc{F}\alpha : \textsf{e}\textsc{F}\alpha/\textsf{e}\textsc{F}\alpha$, and $\textsf{d}\textsc{F}\alpha \in \Gamma(X)$, and $X\not\perp \textsf{e}\textsc{F}\alpha$, then $\textsf{e}\textsc{F}\alpha \in \Gamma(X)$. 




\Large 


%$\Gamma, \langle \phi \land \lnot \psi \rangle_{c,e}, \langle (\alpha \land \lnot \beta) \vartriangleright_{i} (\phi \land \lnot \psi) \rangle_{e} \nc \langle (\alpha \land \lnot \beta)  \leftpitchfork (\phi \land \lnot \psi) \rangle_{e}$
%\UnaryInf$\Gamma \fCenter B \to C$

%Introduction Rule for  `$\leftpitchfork$' or `best explains'
%\begin{prooftree}
%\def\fCenter{\mbox{\ $\nc$\ }}
%\AxiomC{$\Gamma \nc \langle \phi \land \lnot \psi \rangle_{c,e}$}
%\AxiomC{$\Gamma \nc \langle (\alpha \land \lnot \beta) \medtriangleright_{i} (\phi \land \lnot \psi) \rangle_{e}$}
%\RightLabel{  \hspace{5mm}$\leftpitchfork$-Intro}
%\BinaryInf$\Gamma \fCenter \langle (\alpha \land \lnot \beta)  \leftpitchfork (\phi \land \lnot \psi) \rangle_{e}$

%\end{prooftree}\

\begin{prooftree}
\def\fCenter{\mbox{\ $\nc$\ }}
\AxiomC{$\Gamma \nc \langle \psi \rangle_{c,e}$}
\AxiomC{$\Gamma \nc \langle \phi \medtriangleright_{i} \psi \rangle_{e}$}
\RightLabel{  \hspace{5mm}$\leftpitchfork$-Intro}
\BinaryInf$\Gamma \fCenter \langle \phi  \leftpitchfork \psi \rangle_{e}$

\end{prooftree}\


\begin{prooftree}
\def\fCenter{\ \nc\ }
\Axiom$\Gamma \fCenter \langle \phi \leftpitchfork \psi \rangle_{c,e}$
\RightLabel{\hspace{5mm}  $\leftpitchfork$-Elim$_{1}$}
\UnaryInf$\Gamma \fCenter \langle \phi \rangle_{e}$
\end{prooftree}

\begin{prooftree}
\def\fCenter{\ \nc\ }
\Axiom$\Gamma \fCenter \langle \phi \leftpitchfork \psi \rangle_{c,e}$
\RightLabel{\hspace{5mm}$\leftpitchfork$-Elim$_{2}$}
\UnaryInf$\Gamma \fCenter \langle \lnot \phi \boxright \lnot\psi \rangle_{c,e}$
\end{prooftree}

\begin{prooftree}
\def\fCenter{\ \nc\ }
\Axiom$\Gamma \fCenter \langle \phi \leftpitchfork \psi \rangle_{c,e}$
\RightLabel{\hspace{5mm}$\leftpitchfork$-Elim$_{2}$}
\UnaryInf$\Gamma \fCenter \langle \lnot \phi \boxright \lnot\psi \rangle_{c,e}$
\end{prooftree}

Weak Asymmetry of $\leftpitchfork$:

\begin{prooftree}
\def\fCenter{\ \nc\ }
\Axiom$\Gamma \fCenter \langle \phi \leftpitchfork \psi \rangle_{c,e}$
\def\fCenter{\ \not\nc\ }
\UnaryInf$\Gamma \fCenter \langle \psi \leftpitchfork \phi \rangle_{e}$
\end{prooftree}

Strong Asymmetry of $\leftpitchfork$:

\begin{prooftree}
\def\fCenter{\ \nc\ }
\Axiom$\Gamma \fCenter \langle \phi \leftpitchfork \psi \rangle_{c,e}$
\UnaryInf$\Gamma \fCenter \neg\langle \psi \leftpitchfork \phi \rangle_{e}$
\end{prooftree}

This follows from $\leftpitchfork$-Intro and $\leftpitchfork$-Elim$_{1}$:
\begin{prooftree}
\def\fCenter{\ \nc\ }
\AxiomC{$\Gamma\fCenter \langle \phi \leftpitchfork \psi \rangle_{c,e}$}
\UnaryInfC{$\Gamma, \langle \psi \rangle_{c,e} \nc \langle \phi \rangle_{e}$}
\end{prooftree}

$A \not\nc B $

%\begin{prooftree}
%\def\fCenter{\ \nc\ }
%\Axiom$\Gamma, \langle \phi \land \lnot \psi \rangle_{c,e} \fCenter \langle (\alpha \land \lnot \beta) \vartriangleright_{i} (\phi \land \lnot \psi) \rangle_{e}$
%\UnaryInf$\Gamma, \langle \phi \land \lnot \psi \rangle_{c,e} \fCenter \langle (\alpha \land \lnot \beta) \leftpitchfork (\phi \land \lnot \psi) \rangle_{e}$
%\end{prooftree}


%Elimination rules for  `$\leftpitchfork$' or `best explains'
%\begin{prooftree}
%\def\fCenter{\ \nc\ }
%\Axiom$\Gamma \fCenter \langle (\alpha \land \lnot \beta) \leftpitchfork (\phi \land \lnot \psi) \rangle_{c,e}$
%\RightLabel{\hspace{5mm}  $\leftpitchfork$-Elim$_{1}$}
%\UnaryInf$\Gamma \fCenter \langle \alpha \land \lnot \beta \rangle_{e}$
%\end{prooftree}


%\begin{prooftree}
%\def\fCenter{\ \nc\ }
%\Axiom$\Gamma \fCenter \langle (\alpha \land \lnot \beta) \leftpitchfork (\phi \land \lnot \psi) \rangle_{c,e}$
%\UnaryInf$\Gamma \fCenter \langle \phi \land \lnot \psi \rangle_{c,e}$
%\end{prooftree}

%\begin{prooftree}
%\def\fCenter{\ \nc\ }
%\Axiom$\Gamma \fCenter \langle (\alpha \land \lnot \beta) \leftpitchfork (\phi \land \lnot \psi) \rangle_{c,e}$
%\RightLabel{\hspace{5mm}$\leftpitchfork$-Elim$_{2}$}
%\UnaryInf$\Gamma \fCenter \langle (\lnot \alpha \land  \beta) \boxright (\lnot \phi \land  \psi) \rangle_{c,e}$
%\end{prooftree}




%\begin{prooftree}
%\def\fCenter{\ \nc\ }
%\AxiomC{$\Gamma \nc \langle (\alpha \land \lnot \beta) \leftpitchfork (\phi \land \lnot \psi) \rangle_{c,e}$}
%\UnaryInfC{$\Gamma, \langle \phi \land \lnot \psi \rangle_{c,e} \nc \langle \alpha \land \lnot \beta \rangle_{e}$}
%\end{prooftree}


%Harmony: The basic idea of Harmony is that the rules of inference which govern a logical operator should fit with — ‘make sense’ in terms of — each other: the E-rules should not enable one to over-extend — derive more from a statement S in which that operator is dominant than is justified by the premises which sanction S via the I-rules. 

%The idea behind harmony is that the elimination-rule should allow one to infer all and only what is justified by the meaning conferred by the introduction-rule




\end{document}