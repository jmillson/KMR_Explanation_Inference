\documentclass{article}
\usepackage[top=1in,bottom=1in,left=1.25in,right=1.25in]{geometry}
\usepackage{setspace}
\usepackage{amsmath}
\usepackage{amsthm}
\usepackage{amssymb}
\usepackage{stmaryrd}
\usepackage{natbib}
\usepackage{enumitem}
\usepackage{times}
\usepackage{graphicx}
\usepackage{latexsym}
\usepackage{bussproofs}
\usepackage{MnSymbol}
\usepackage{wasysym}
\usepackage{turnstile}



\DeclareSymbolFont{symbolsC}{U}{ntxsyc}{m}{n}
\SetSymbolFont{symbolsC}{bold}{U}{ntxsyc}{n}{b}
\DeclareMathSymbol{\multimapdotbothA}{\mathrel}{symbolsC}{23}
\DeclareMathSymbol{\boxright}{\mathrel}{symbolsC}{128}

\usepackage[hidelinks]{hyperref}
%\usepackage{lingmacros}
\hypersetup{
    colorlinks=false,
    pdfborder={0 0 0},
}
\newcommand\pref[1]{(\ref{#1})}
%\usepackage{calc}
%\usepackage{covington}
%\usepackage{diagbox}
%\usepackage{fixltx2e}
%\usepackage{tikz}
%\usetikzlibrary{calc}

%\newcommand{\hcancel}[5]{%
%    \tikz[baseline=(tocancel.base)]{
%        \node[inner sep=0pt,outer sep=0pt] (tocancel) {#1};
%        \draw[red] ($(tocancel.north west)$) -- ($(tocancel.south east)$);}}%

%\DeclareFontEncoding{LGR}{}{}
%\DeclareTextSymbol{\~}{LGR}{126}

%\providecommand{\tabularnewline}{\\}
%\newcommand*{\equationautorefname}[1]{\@gobble}
\setcitestyle{aysep={}, notesep=:}

%opening
\title{Inferentialism for Best Explanations}
\author{Kareem Kalifa, Jared Millson, Mark Risjord}
\date{}
\raggedbottom


%\newtheorem*{pppq}{Basic Proto-Polar Queries}
%\newtheorem*{ppq}{Proto-Polar ISQs}
%\newtheorem*{rsq}{Reason-Seeking Queries}
%\newcounter{acounter}
%\newcounter{qcounter}
%\newcounter{dcounter}
\newcommand{\nc}{\,\mid\!\sim\,}

\begin{document}
\maketitle
\setlength{\parindent}{1cm}

%Let $\alpha, \beta, \phi, \psi$ stand for sentences of a first-order language $\mathcal{L}$. The formula $\langle \alpha \rangle_{s}^{X}$ says that a conversational participant, \textit{X}, holds some normative attitude(s), \textit{s}, toward the assertion of  sentence $\alpha$. The variable \textit{s} takes the value \textit{c} when X is committed to assert $\alpha$ and  \textit{e} when X is entitled to assert $\alpha$. We will refer to these formulas as \textit{normative assignments}. Let \textit{A, B, C,} range over normative assignments and $\Gamma, \Delta, \Theta$ range over sets of normative assignments. The set of all normative assignments attributed to X is X's scorecard, $\Gamma^{X}$.  In what follows, we assume all inferences are  undertaken by either a single agent or by different agents on the basis of a common ground of normative assignments, in which case $\Gamma$ can be thought of as representing the conversational common ground or context. In either case, the specification of participants is unnecessary and therefore has been omitted.
%
%Sequent calculus versions of formal proof theory can now be applied to explain the constraints on scorecards. $\Delta$ $\vdash$ $\Theta$ is a structural constraint on scorecards such that a scorecard satisfies it if and only if it is not the case that the it contains every attribution in $\Delta$ and none in $\Theta$. General constraints on scorecards are analogous to the normal structural rules for the consequence relation, i.e., \textit{reflexivity}, \textit{weakening}, and \textit{cut}.
%
%A special case of the constraint is the form $\Gamma\vdash$. \, A scorecard is said to violate this constraint when it contains every assignment in $\Gamma$, and respects it when there is some assignment in $\Gamma$ that it avoids. $\Gamma$ $\vdash$ says that the assignments in $\Gamma$ are jointly incompatible. For clarity, $\Gamma\vdash$ will be written as $\perp\Gamma$ and where $\Gamma$ = \{A, B\}, $\perp\Gamma$ can be written as A $\perp$ B.
%%This is the primitive notion of incompatibility that the system uses to build up incompatibility relations among specific normative statuses.
%
%We now use the standard constraints of $\vdash$ to define a nonmonotonic structure over normative assignments. A set of normative assignments nonmonotonically  $\Gamma\nc A$ \textit{iff} $\Gamma\vdash A$ and if $A \perp B$, then $ B \not\in \{C: \Gamma\vdash C\}$. By definition, the nonmonotonic consequence relation $\nc$ satisfies \textit{reflexivity} and \textit{supraclassicality}, but we stipulate that it also satisfies \textit{cautious monotonicity}:
%
%\begin{prooftree}
%\def\fCenter{\mbox{\ $\nc$\ }}
%\AxiomC{$\Gamma \nc A$}
%\AxiomC{$\Gamma \nc B$}
%\RightLabel{  \hspace{5mm} Caut. Mono.}
%\BinaryInf$\Gamma, A \fCenter B$
%
%\end{prooftree}\
%
%Let `$\medtriangleright$' be a binary operator expressing a thick-bundle relation between sentences. 
%%Let $W$ be the set of all normative assignments belonging to participants of a conversation and let $\{\alpha : \Delta \vdash \alpha\}$ represent the closure of  a set of assignments $\Delta$ under $\vdash$. \hspace{.5mm}Our system has a single default theory $\langle W,D\rangle$ where $D$ is a set of defaults composed of instances of the following form with particular action-types substituted for F and particular agents for $\alpha, \beta$.
%%
%%$$\frac{\textsf{d}\textsc{F}\alpha : \textsf{e}\textsc{F}\alpha}{\textsf{e}\textsc{F}\alpha}$$
%%
%%%Given a pair of sets of assignments ($X, Y$), a default is trigger by ($X, Y$) \emph{iff} $X\vdash \textsf{d}\textsc{F}\alpha$ and $Cn(Y)\not\perp \textsf{e}\textsc{F}\alpha$. 
%%Rules of this form say that if an agent has F-ed and there is no assignment incompatible with its entitlement, then the agent is entitled to F. Extensions of $\langle W,D\rangle$ are defined by fixed point construction. For any set of normative assignments, $X$, let $\Gamma(X)$ be the least closed set that includes $W$ and satisfies the following condition: If $\textsf{d}\textsc{F}\alpha : \textsf{e}\textsc{F}\alpha/\textsf{e}\textsc{F}\alpha$, and $\textsf{d}\textsc{F}\alpha \in \Gamma(X)$, and $X\not\perp \textsf{e}\textsc{F}\alpha$, then $\textsf{e}\textsc{F}\alpha \in \Gamma(X)$. 




\Large 


%$\Gamma, \langle \phi \land \lnot \psi \rangle_{c,e}, \langle (\alpha \land \lnot \beta) \vartriangleright_{i} (\phi \land \lnot \psi) \rangle_{e} \nc \langle (\alpha \land \lnot \beta)  \leftpitchfork (\phi \land \lnot \psi) \rangle_{e}$
%\UnaryInf$\Gamma \fCenter B \to C$

%Introduction Rule for  `$\leftpitchfork$' or `best explains'
%\begin{prooftree}
%\def\fCenter{\mbox{\ $\nc$\ }}
%\AxiomC{$\Gamma \nc \langle \phi \land \lnot \psi \rangle_{c,e}$}
%\AxiomC{$\Gamma \nc \langle (\alpha \land \lnot \beta) \medtriangleright_{i} (\phi \land \lnot \psi) \rangle_{e}$}
%\RightLabel{  \hspace{5mm}$\leftpitchfork$-Intro}
%\BinaryInf$\Gamma \fCenter \langle (\alpha \land \lnot \beta)  \leftpitchfork (\phi \land \lnot \psi) \rangle_{e}$

%\end{prooftree}\
Mark/Jared rules for 'Best explains' 1/2/16
\vspace{1cm}

\underline{Preliminaries:}
\begin{itemize}

\item Let $\alpha, \beta, \phi, \psi$ stand for sentences of a propositional logic, and $\Gamma, \Delta, etc.$ stand for sets of such sentences.

\item $\phi > \psi$ = `If $\phi$ were the case, then $\psi$ would be the case'

\item $\phi \rhd \psi$ = `$\phi$ potentially explains $\psi$'

\item $\phi \RHD \psi$ = `$\phi$ best explains $\psi$'

\item Let $\succeq$ be a partial order on sentences, i.e. an reflexive, transitive relation between sentences.

\item $\phi \succeq_{\psi} \delta$ = `$\phi$  is at least as good an explanation of $\psi$ as $\delta$'

\item $ \phi\succeq_{\psi} $ = `$\phi$  is the best explanation of $\psi$ (relative to a context)', i.e. $\{\delta | \delta \in \Gamma, \delta \rhd \psi\}$, if $\delta\succeq_\psi \phi$ then $\phi \succeq_\psi \delta$.

\item Let $\vdash$ be a classical consequence relation, i.e. $\vdash$ satisfies reflexivity, cut, and weakening (monotonicity).

\item Let $\nc$ be a non-monotonic consequence relation that satisfies reflexivity, cut, cautious monotonicity, and supraclassicality.

\end{itemize}
Structural rules on `$\rhd$', meta-logical versions of those borrowed from Chellas, via Aliseda.


\begin{prooftree}
%\def\fCenter{\ \dashv\ }
\AxiomC{$\phi \dashv\vdash \psi$}
\RightLabel{\hspace{5mm}RCEA}
\UnaryInfC{$\phi \rhd \chi \dashv\vdash \psi \rhd \chi$}
\end{prooftree}

\begin{prooftree}
%\def\fCenter{\ \dashv\ }
\AxiomC{$\phi, \psi \vdash \chi$}
\RightLabel{\hspace{5mm}RCR}
\UnaryInfC{$\gamma \rhd \phi,\gamma \rhd \psi  \vdash \gamma \rhd \chi$}
\end{prooftree}

\newpage
Some Rules we came up with:
\begin{prooftree}
\def\fCenter{\mbox{\ $\nc$\ }}
\AxiomC{$\Gamma \nc \phi > \psi$}
\AxiomC{$\Gamma \nc\psi$}
\RightLabel{  \hspace{5mm}$\rhd$-R}
\BinaryInf$\Gamma \fCenter \phi  \rhd \psi $

\end{prooftree}\

\begin{prooftree}
\def\fCenter{\ \nc\ }
\AxiomC{$\Gamma \nc \phi \rhd \psi$}
\AxiomC{$\Gamma \nc \phi\succeq_{\psi} $}
\RightLabel{\hspace{5mm}  $\RHD$-R}
\BinaryInf$\Gamma \fCenter \phi  \RHD \psi$
\end{prooftree}


\begin{prooftree}
\def\fCenter{\ \nc\ }
\AxiomC{$\Gamma ,\phi \RHD\psi\nc  \psi$}
\AxiomC{$\Gamma ,\phi \RHD\psi, \phi \nc \theta$}
\RightLabel{\hspace{5mm}  $\RHD$-L}
\BinaryInf$\Gamma,  \phi  \RHD \psi \fCenter \theta$
\end{prooftree}



\begin{prooftree}
\def\fCenter{\ \nc\ }
\AxiomC{$\Gamma\nc  \phi \RHD\psi$}
\AxiomC{$\Gamma \nc \psi$}
\RightLabel{\hspace{5mm}  $\RHD$-L*}
\BinaryInf$\Gamma\fCenter \phi $
\end{prooftree}


Note: R-rules are introductions and L-rules are eliminations. $\RHD$-L is equivalent to $\RHD$-L* and both are IBE/backwards MP.
\vspace{1cm}

\underline{Some problems:}
\begin{itemize}

\item The including of the explanadum in $\rhd$-R was motivated by the need to rule out subjunctive conditionals with non-actual consequents. But in doing so, it makes the inclusion of the explanadum in IBE, i.e. $\RHD$-L redundant. I think we should drop the explanadum in $\rhd$-R to avoid this. It opens up the possibility for possible explanations of non-actual explanada, but I think that's OK. The question is whether the acceptance of subjunctive conditional is sufficient to warrant the corresponding possibly-explains claim.

\item There is no proper elimination rule for $\rhd$, even though $\RHD$-R does eliminate it.

\item The inclusion of the subjunctive conditional was in part motivated by my buddy Preston Stovall's dissertation, where he gave a proof-theoretic account of subjunctive conditionals. For non-counterfactual subjunctive conditionals, his rules are as follows (we discussed his `update function' but I later realized that this was only needed for counterfactual subjunctives):

\begin{prooftree}
\def\fCenter{\ \nc\ }
\AxiomC{$\Gamma, \phi \nc  \psi$}
\RightLabel{\hspace{5mm}  $>$-R}
\UnaryInfC{$\Gamma \fCenter \phi > \psi$}
\end{prooftree}

\begin{prooftree}
\def\fCenter{\ \nc\ }
\AxiomC{$\Gamma \nc  \phi$}
\AxiomC{$\Gamma, \psi \nc  \chi$}
\AxiomC{$\Gamma, \psi \nc  \phi$}
\RightLabel{\hspace{5mm}  $>$-L}
\TrinaryInfC{$\Gamma,  \phi > \psi \fCenter \chi$}
\end{prooftree}

If these rules are correct, then we would have something really cool: a link from material, non-monotonic inferences made explicit by subjunctive conditionals, to explains-claims.

\item Unfortunately, I don't think his rules will work, and here's why: Non-monotonic inferences cannot be made explicit by defeasible conditionals in the way that deductive inferences can be made explicit by the material conditional. The reason is that the deduction theorem fails for non-monotonic consequence relations. That is, $\Gamma, \phi \nc  \psi \not\leftrightarrow \Gamma \nc  \phi \rightsquigarrow\psi$, where $\phi \rightsquigarrow\psi$ is supposed to be a defeasible conditional that makes explicit the non-monotonic consequence. The problem is that so long as $\nc$ is reflexive and transitive, $ \rightsquigarrow$ will be monotonic. This basically puts the nails to Preston's proof theory. (I called him and confirmed the problem).

\item Here's a possible solution: what if the basic material inferences are both non-monotonic and non-transitive? It's pretty bizarre to think of a consequence relation as non-transitive, but I have actually found an inferentialist semantics for classical logic that uses a non-transitive consequence!! Check it out: \url{https://www.dropbox.com/s/rp5dya2131hx8bc/RIPPAF.1.pdf?dl=0} (Ripley is following in the bilateralist inferentialist tradition of Restall, whose work was very important to the formal theory in my dissertation). 

\item So here's a new possibility. Start with a monotonic but non-transitive consequence relation for classical prop logic. Define a non-monotonic consequence over a fragment of the language. Then use that to introduce a non-transitive defeasible conditional that, I think, will behave just as we want $\rhd$, the 'possibly explains' connective to behave. Update coming soon!




\end{itemize}


\end{document}