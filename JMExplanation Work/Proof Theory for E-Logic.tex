\documentclass{article}                     % onecolumn (standard format)
%\documentclass[smallcondensed]{svjour3}     % onecolumn (ditto)
%\documentclass[smallextended]{svjour3}       % onecolumn (second format)
%\documentclass[twocolumn]{svjour3}          % twocolumn
%

\usepackage{geometry}
\usepackage{graphicx}
\usepackage{amsmath}
\usepackage{amsthm}
%\usepackage{mathptmx}
\usepackage{stmaryrd}
\usepackage{enumitem}
\usepackage{times}
\usepackage{graphicx}
\usepackage{latexsym}
\usepackage{bussproofs}
\usepackage{pgf}
\usepackage{adjustbox}
\usepackage{multirow}
\usepackage{array}
\usepackage{tikz}
\usepackage{xcolor}
\usepackage{ushort}
\usepackage{soul}
\usepackage[autostyle]{csquotes}
\usepackage[doi=false,isbn=false,url=false,style=chicago-authordate,natbib=true]{biblatex}

%\usepackage{fonttable}

\addbibresource{KMR_Master.bib}

\newtheorem{theorem}{Theorem}[section]
\newtheorem{corollary}{Corollary}[theorem]
\newtheorem{lemma}[theorem]{Lemma}
\theoremstyle{definition}
\newtheorem{definition}{Definition}[section]


\DeclareMathSymbol{\Gamma}{\mathalpha}{operators}{0}
\DeclareMathSymbol{\Delta}{\mathalpha}{operators}{1}
\DeclareMathSymbol{\Theta}{\mathalpha}{operators}{2}
\DeclareMathSymbol{\Lambda}{\mathalpha}{operators}{3}
\DeclareMathSymbol{\Xi}{\mathalpha}{operators}{4}
\DeclareMathSymbol{\Pi}{\mathalpha}{operators}{5}
\DeclareMathSymbol{\Sigma}{\mathalpha}{operators}{6}
\DeclareMathSymbol{\Upsilon}{\mathalpha}{operators}{7}
\DeclareMathSymbol{\Phi}{\mathalpha}{operators}{8}
\DeclareMathSymbol{\Psi}{\mathalpha}{operators}{9}
\DeclareMathSymbol{\Omega}{\mathalpha}{operators}{10}


\DeclareFontFamily{U} {MnSymbolA}{}

\DeclareFontShape{U}{MnSymbolA}{m}{n}{
  <-6> MnSymbolA5
  <6-7> MnSymbolA6
  <7-8> MnSymbolA7
  <8-9> MnSymbolA8
  <9-10> MnSymbolA9
  <10-12> MnSymbolA10
  <12-> MnSymbolA12}{}
\DeclareFontShape{U}{MnSymbolA}{b}{n}{
  <-6> MnSymbolA-Bold5
  <6-7> MnSymbolA-Bold6
  <7-8> MnSymbolA-Bold7
  <8-9> MnSymbolA-Bold8
  <9-10> MnSymbolA-Bold9
  <10-12> MnSymbolA-Bold10
  <12-> MnSymbolA-Bold12}{}

\DeclareSymbolFont{MnSyA}{U}{MnSymbolA}{m}{n}
\DeclareMathSymbol{\twoheaduparrow}{\mathop}{MnSyA}{25}
\DeclareMathSymbol{\twoheadrightarrow}{\mathop}{MnSyA}{24}

\makeatletter

\usetikzlibrary{shapes}
\usetikzlibrary{fit,shapes.misc}

\newcommand\marktopleft[1]{%
	\tikz[overlay,remember picture] 
	\node (marker-#1-a) at (3cm, 1mm) {};%
}
\newcommand\markbottomright[1]{%
	\tikz[overlay,remember picture] 
	\node (marker-#1-b) at (-3cm,7mm) {};%
	\tikz[overlay,remember picture,thick,inner sep=3pt]
	\node[draw,rounded rectangle,fit=(marker-#1-a.center) (marker-#1-b.center)] {};%
}





\newcommand{\ee}{\twoheadrightarrow}

% % % % % % % % % % % % % % % % Footnote Command % % % % % % % % % % % % %
\usepackage{refcount}% http://ctan.org/pkg/refcount
\newcounter{fncntr}
\newcommand{\fnmark}[1]{\refstepcounter{fncntr}\label{#1}\footnotemark[\getrefnumber{#1}]}
\newcommand{\fntext}[2]{\footnotetext[\getrefnumber{#1}]{#2}}

% % % % % % % % % % % % % % % Internal Commands NMC% % % % % % % % % % % % %
\newcommand{\raisemath}[1]{\mathpalette{\raisem@th{#1}}}
\newcommand{\raisem@th}[3]{\raisebox{#1}{$#2#3$}}

\newcommand{\uuparrow}{% 
	\raisebox{.165ex}{\clipbox{0pt .6pt 0pt 0pt}{$\uparrow$}}
}
\newcommand{\tuuparrow}{% 
	\raisebox{.165ex}{\clipbox{0pt 1pt 0pt 0pt}{$\scriptscriptstyle\uparrow$}}
}
\newcommand{\muparrow}{% 
	\raisebox{.05ex}{\clipbox{0pt .65pt 0pt 0pt}{$\scriptstyle\uparrow$}}
}
\newcommand{\Uuparrow}{% 
	\raisebox{.2ex}{\clipbox{0pt .15pt 0pt 0pt}{$\Uparrow$}}
}
\newcommand{\thuarrow}{% 
	\raisebox{.05ex}{\clipbox{0pt .8pt 0pt 0pt}{$\twoheaduparrow$}}
}

% % % % % COMMANDS FOR NON-MONOTONIC CONSEQUENCES % % % % % % % % %
\newcommand{\nms}{%
	\mathbin{\mathpalette\@nms\expandafter}
}
\newcommand{\@nms}{\mid\joinrel\mkern-.5mu\sim}


\newcommand{\nmc}{%
	\mathbin{\mathpalette\nm@\expandafter}
}
\newcommand{\nm@}{\mid\joinrel\mkern-.5mu\sim\mkern-3mu}

\newcommand{\qmc}[1]{\mathrel{
		\mathchoice
		{\normalsize\hspace{.4mm}\nms^{\mkern-18mu\scriptsize\uuparrow#1}\hspace{-.7mm}}
		{\normalsize\hspace{.4mm}\nms^{\mkern-18mu\scriptsize\uuparrow#1}\hspace{-.7mm}}
		{\footnotesize\hspace{.4mm}\nms^{\mkern-13mu\tiny\uuparrow#1}}
		{\scriptsize\nms^{\mkern-10mu\tiny\tuuparrow#1}}
	}
}

\newcommand{\mqmc}{\mathrel{
		\mathchoice
		{\hspace{.4mm}\nms^{\mkern-18mu\scriptsize\uuparrow}\hspace{.6mm}}
		{\hspace{.4mm}\nms^{\mkern-18mu\scriptsize\uuparrow}\hspace{.6mm}}
		{\footnotesize\hspace{.4mm}\nms^{\mkern-11mu\tiny\uuparrow}\hspace{.6mm}}
		{\scriptsize\nms^{\mkern-10mu\tiny\tuuparrow}}
	}
}

\newcommand{\mrc}[1]{\mathbin{
		\mathchoice
		{\normalsize\hspace{.5mm}\nms^{\mkern-19mu\scriptsize\Uuparrow#1}\hspace{-.5mm}}
		{\normalsize\hspace{.5mm}\nms^{\mkern-19mu\scriptsize\Uuparrow#1}\hspace{-.5mm}}
		{\footnotesize\hspace{.5mm}\nms^{\mkern-13.5mu\fontsize{5.5}{0}\Uuparrow#1}}
		{\scriptsize\nms^{\mkern-10mu\tiny\Uuparrow#1}}
	}
}

\newcommand{\smc}{\mathbin{
		\mathchoice
		{\hspace{.4mm}\nms^{\mkern-17mu\scriptsize\thuarrow}\hspace{.6mm}}
		{\hspace{.4mm}\nms^{\mkern-17mu\scriptsize\thuarrow}\hspace{.6mm}}
		{\footnotesize\hspace{.4mm}\nms^{\mkern-11mu\tiny\thuarrow}\hspace{.6mm}}
		{\scriptsize\nms^{\mkern-10mu\tiny\thuarrow}}
	}
}
\newcommand{\nnmc}{\not\mkern-4mu\nmc}
\newcommand{\nsmc}{\not\mkern-3mu\smc}
\newcommand{\nmrc}{\not\mkern-3mu\mrc}
\newcommand{\nmqmc}{\not\mkern1mu\mqmc}
\newcommand{\nqmc}{\not\mkern1mu\qmc}

\newcommand{\nme}{%
	\mathbin{\mathpalette\@nme\expandafter}
}
\newcommand{\@nme}{{\mid\joinrel\mkern-.5mu\sim\mkern-2mu}_{e}\mkern3mu}


%\newcommand{\nme}{\nms\mkern-7mu_{e}\mkern2mu}
\newcommand{\qme}{{\qmc\mkern-2mu}_{e}\mkern3mu}
\newcommand{\mqme}{{\mqme\mkern-2mu}_{e}\mkern3mu}
\newcommand{\sme}{{\smc\mkern-2mu}_{e}\mkern3mu}

% % % % % % % % % % %Commands for Material Incoherence% % % % % % % % % % % %

\newcommand{\bigperpp}{%
	\mathop{\mathpalette\bigp@rpp\relax}%
	\displaylimits
}
\newcommand{\bigp@rpp}[2]{%
	\vcenter{
		\m@th\hbox{\scalebox{\ifx#1\displaystyle1.15\else1.15\fi}{$#1\perp$}}
	}%
}
\newcommand{\bigperp}{\raisemath{.5pt}{\bigperpp}}

%% % % % % % Degree Command % % % % % % % % % % % % % % % % % % % %
\newcommand{\degree}{\ensuremath{^\circ}}

%%%%%%%%%%%%%Author Comments%%%%%%%%%%%%%%%%%
 \newcommand{\kk}[1]{\textcolor{red}{$^{\textrm{KK}}${#1}}}
 \newcommand{\jm}[1]{\textcolor{blue}{$^{\textrm{JM}}${#1}}}
 \newcommand{\mr}[1]{\textcolor{green}{$^{\textrm{MR}}${#1}}}


\newcolumntype{L}[1]{>{\raggedright\let\newline\\\arraybackslash\hspace{0pt}}m{#1}}
\newcolumntype{C}[1]{>{\centering\let\newline\\\arraybackslash\hspace{0pt}}m{#1}}
\newcolumntype{R}[1]{>{\raggedleft\let\newline\\\arraybackslash\hspace{0pt}}m{#1}}



\makeatother





%\renewcommand{\@cite}[1]{#1}
%\usepackage[natbibapa]{apacite}
\usepackage[hang,flushmargin]{footmisc} 
\usepackage[hidelinks]{hyperref}
%\usepackage{lingmacros}
%\hypersetup{
%    colorlinks=false,
%    pdfborder={0 0 0},
%}


\begin{document}
\sloppy
\raggedbottom
\title{\vspace{-3cm}Proof Theoretic Semantics for Questions}

\raggedbottom

\maketitle

In addition to $ \mathcal{L}_e $, we construct another language, called $ \mathcal{L}_Q $ which has two categories of well-formed expressions: \textit{d-wffs} and \textit{e-wffs}, that is, claims and questions. (For simplicity, we drop the '\textit{e}' notation from our turnstile.) The syntax of $ \mathcal{L}_Q $ is just that of $ \mathcal{L}_e $ plus the following: ?, \{, \}, and the comma. D-wffs are just sentences of $ \mathcal{L}_e $.  E-wffs, on the other hand, are expressions of the following form:  $ ?\{A_1,\ldots, A_n\}  $ where $ n > 1 $ and $A_1\ldots A_n  $ are pairwise syntactically distinct d-wffs of $ \mathcal{L}_e $. If $ ?\{A_1,\ldots, A_n\} $ is a question, then each of the d-wffs $ A_1\ldots A_n $ is a possible direct answer to the question, and these are the only possible direct answers. Not that any question in $ \mathcal{L}_Q $ has at least two possible direct answers and that the set of those answers is always finite. We must stress that e-wffs are NOT sets of d-wffs, but rather singular expressions of a strictly defined form. We can paraphrase e-wffs as aksing \textit{Which of the following claims am I entitled to: $ A_1,\ldots, A_n $? }.

%e.g. $ ?\{A, B\} \neq \{B, A\} $.


Here are possible rules for e-wffs:

\vspace{.3cm}

\centering\textbf{Right rule for $ ?\{A_1,\ldots, A_n\} $}
\begin{prooftree}
\def\fCenter{\ \nmc\ }
\AxiomC{$ \Gamma,\, A_1\nmc B \ldots \Gamma,\, A_n\nmc B $\hspace{-.7cm}}
\RightLabel{\hspace{5mm}  where $ \Gamma, B \nnmc A_1 \ldots \Gamma, B \nnmc A_n $}
\UnaryInf $\Gamma,\, B \fCenter ?\{A_1,\ldots, A_n\} $
\end{prooftree}

\vspace{.3cm}

\centering\textbf{Possible left rule for $ ?\{A_1,\ldots, A_n\} $}
\begin{prooftree}
\def\fCenter{\ \nmc \ }
\AxiomC{\hspace{-2.5cm}$\Gamma \nmc A_i$}
\RightLabel{\hspace{5mm}  where $ 1 \leq i \leq n $}
\UnaryInf$\Gamma, \,?\{A_1,\ldots, A_n\} \fCenter \bigperp $
\end{prooftree}

%\centering\textbf{Possible left rule for $ ?\{A_1,\ldots, A_n\} $}
%\begin{prooftree}
%\def\fCenter{\ \nmc \ }
%\AxiomC{\hspace{-.1cm}$\Gamma \nmc A_i$}
%\AxiomC{\hspace{-.1cm}$\Gamma,\,A_i \nmc B$}
%\RightLabel{\hspace{5mm}  where $ 1 \leq i \leq n $}
%\BinaryInf$\Gamma, \,?\{A_1,\ldots, A_n\} \fCenter B $
%\end{prooftree}
\vspace{.5cm}

\textbf{(Negative) Partial Answer}

\begin{prooftree}
\def\fCenter{\ \nmc\ }
\AxiomC{$ \Gamma, \,B,\,A_i \nmc \bigperp$}
\RightLabel{\hspace{5mm}  where $ n > 2 $ and $ 1 \leq i \leq n $}
\UnaryInf$\Gamma, \,B,\,?\{A_1,\ldots, A_n\} \fCenter ?\{A_1,\ldots, A_{n-1}\} $
\end{prooftree}

\flushleft
\vspace{.5cm}
We can give the semantics for a non-factive why-question, \textit{Why B?}, as follows: $$ ?\{A_1\ee B,\ldots, A_{n}\ee B\} $$
Once $ \mathcal{L}_e $ is extended to include traditional logical operators, we can give the semantics for \textit{factive} why questions, e.g. \textit{Why B?}, as follows:

$$ ?\{(A_1\ee B)\wedge B,\ldots, (A_{n}\ee B)\wedge B\} $$

Notice that once one has an answer to this question, one is entitled to the explanans.




\vspace{10cm}

\begin{prooftree}
\def\fCenter{\ \nme\ }
\AxiomC{$ \Gamma,\, A\, \sme B $\hspace{-.3cm}}
\RightLabel{\hspace{5mm}  EER}
\UnaryInf$\Gamma \fCenter A \mathbin{\ee} B$
\end{prooftree}

\begin{prooftree}
\def\fCenter{\ \sme \ }
\AxiomC{\hspace{-.8cm}$\Gamma \nme A \ee B$}
\RightLabel{\hspace{5mm}  EES}
\UnaryInf$\Gamma, \,A \, \fCenter B $
\end{prooftree}

\begin{prooftree}
\def\fCenter{\ \nme\ }
\AxiomC{$ \Gamma, \,A \sme B$}
\AxiomC{$\Gamma \fCenter \hspace{-1ex}  B $}
\RightLabel{\hspace{5mm}  EEL}
\BinaryInf$\Gamma, B, A \ee B\fCenter \hspace{-1ex}  A$
\end{prooftree}




\flushleft

\noindent\textbf{Quantified Material Consequence (QMC):}\label{QMC}
\begin{equation}
	\Gamma, A \qmc{W} B \Longleftrightarrow_{df}
	\begin{cases}\nonumber
		1.\,\, W\subseteq\mathcal{P}(\mathcal{L}) & $ and $ \\[3pt] 
		2.\,\, \forall\Delta\in W(\Gamma, A,  \Delta \nmc B ) 
	\end{cases}
\end{equation}

\noindent\textbf{Modally Robust Consequence (MRC):}\label{MRC}
\begin{equation}
	\Gamma, A \mrc{W} B \Longleftrightarrow_{df}
	\begin{cases}\nonumber
		1.\,\, \Gamma, A \qmc{W} B& $ and $ \\[3pt] 
		2.\,\, \forall W'(\Gamma, A \qmc{W'} B \Longrightarrow W' \subseteq W)  & $ and $\\[3pt] 
		%3.\,\, \nqmc[]{\Gamma}{B}{W} & $ and $\\[3pt]  
		3.\,\, \forall\Delta(\Gamma, A, \Delta \mqmc \bigperp \Longrightarrow \Delta \not\in W)
		
	\end{cases}
\end{equation}



\noindent\textbf{Sturdy  Consequence (SC): [Sets]}\label{SCset}
\begin{equation}
\Gamma, \, \Sigma \smc B \Longrightarrow
\begin{cases}\nonumber 
1.\,\, \Gamma, \, \Sigma \mrc{W} B & $ where $ B \not\in \Sigma $, and $ \\[3pt] 
2.\,\, \forall \Theta\,(\Gamma, \Theta \mrc{W'} B \Longrightarrow W\not\subset W' ) & $ where $ B\not\in \Theta  \\[3pt] 
\end{cases}
\end{equation}







\printbibliography


\end{document}
