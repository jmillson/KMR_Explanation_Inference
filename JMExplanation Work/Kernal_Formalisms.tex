\documentclass{article}                     % onecolumn (standard format)
%\documentclass[smallcondensed]{svjour3}     % onecolumn (ditto)
%\documentclass[smallextended]{svjour3}       % onecolumn (second format)
%\documentclass[twocolumn]{svjour3}          % twocolumn
%
\usepackage{geometry}
\usepackage{graphicx}
\usepackage{amsmath}
\usepackage{mathptmx}
\usepackage{stmaryrd}
\usepackage{enumitem}
\usepackage{times}
\usepackage{graphicx}
\usepackage{latexsym}
\usepackage{bussproofs}
\usepackage{pgf}
\usepackage{adjustbox}
\usepackage{xcolor}
\usepackage{ushort}
\usepackage{soul}
\usepackage[autostyle]{csquotes}
\usepackage[doi=false,isbn=false,url=false,style=chicago-authordate,natbib=true]{biblatex}

%\usepackage{fonttable}

\addbibresource{KMR_Master.bib}



\DeclareMathSymbol{\Gamma}{\mathalpha}{operators}{0}
\DeclareMathSymbol{\Delta}{\mathalpha}{operators}{1}
\DeclareMathSymbol{\Theta}{\mathalpha}{operators}{2}
\DeclareMathSymbol{\Lambda}{\mathalpha}{operators}{3}
\DeclareMathSymbol{\Xi}{\mathalpha}{operators}{4}
\DeclareMathSymbol{\Pi}{\mathalpha}{operators}{5}
\DeclareMathSymbol{\Sigma}{\mathalpha}{operators}{6}
\DeclareMathSymbol{\Upsilon}{\mathalpha}{operators}{7}
\DeclareMathSymbol{\Phi}{\mathalpha}{operators}{8}
\DeclareMathSymbol{\Psi}{\mathalpha}{operators}{9}
\DeclareMathSymbol{\Omega}{\mathalpha}{operators}{10}


\DeclareFontFamily{U} {MnSymbolA}{}

\DeclareFontShape{U}{MnSymbolA}{m}{n}{
  <-6> MnSymbolA5
  <6-7> MnSymbolA6
  <7-8> MnSymbolA7
  <8-9> MnSymbolA8
  <9-10> MnSymbolA9
  <10-12> MnSymbolA10
  <12-> MnSymbolA12}{}
\DeclareFontShape{U}{MnSymbolA}{b}{n}{
  <-6> MnSymbolA-Bold5
  <6-7> MnSymbolA-Bold6
  <7-8> MnSymbolA-Bold7
  <8-9> MnSymbolA-Bold8
  <9-10> MnSymbolA-Bold9
  <10-12> MnSymbolA-Bold10
  <12-> MnSymbolA-Bold12}{}

\DeclareSymbolFont{MnSyA}{U}{MnSymbolA}{m}{n}
\DeclareMathSymbol{\twoheaduparrow}{\mathop}{MnSyA}{25}
\DeclareMathSymbol{\twoheadrightarrow}{\mathop}{MnSyA}{24}

\makeatletter

% % % % % % % % % % % % % % % % Footnote Command % % % % % % % % % % % % %
\usepackage{refcount}% http://ctan.org/pkg/refcount
\newcounter{fncntr}
\newcommand{\fnmark}[1]{\refstepcounter{fncntr}\label{#1}\footnotemark[\getrefnumber{#1}]}
\newcommand{\fntext}[2]{\footnotetext[\getrefnumber{#1}]{#2}}

% % % % % % % % % % % % % % % Internal Commands NMC% % % % % % % % % % % % %
\newcommand{\raisemath}[1]{\mathpalette{\raisem@th{#1}}}
\newcommand{\raisem@th}[3]{\raisebox{#1}{$#2#3$}}

\newcommand{\uuparrow}{% 
	\raisebox{.165ex}{\clipbox{0pt .6pt 0pt 0pt}{$\uparrow$}}
}
\newcommand{\tuuparrow}{% 
	\raisebox{.165ex}{\clipbox{0pt 1pt 0pt 0pt}{$\scriptscriptstyle\uparrow$}}
}
\newcommand{\muparrow}{% 
	\raisebox{.05ex}{\clipbox{0pt .65pt 0pt 0pt}{$\scriptstyle\uparrow$}}
}
\newcommand{\Uuparrow}{% 
	\raisebox{.2ex}{\clipbox{0pt .15pt 0pt 0pt}{$\Uparrow$}}
}
\newcommand{\thuarrow}{% 
	\raisebox{.05ex}{\clipbox{0pt .8pt 0pt 0pt}{$\twoheaduparrow$}}
}

% % % % % COMMANDS FOR NON-MONOTONIC CONSEQUENCES % % % % % % % % %
\newcommand{\nms}{%
	\mathbin{\mathpalette\@nms\expandafter}
}
\newcommand{\@nms}{\mid\joinrel\mkern-.5mu\sim}


\newcommand{\nmc}{%
	\mathbin{\mathpalette\nm@\expandafter}
}
\newcommand{\nm@}{\mid\joinrel\mkern-.5mu\sim\mkern-3mu}

\newcommand{\qmc}[1]{\mathrel{
		\mathchoice
		{\normalsize\hspace{.4mm}\nms^{\mkern-18mu\scriptsize\uuparrow#1}\hspace{-.7mm}}
		{\normalsize\hspace{.4mm}\nms^{\mkern-18mu\scriptsize\uuparrow#1}\hspace{-.7mm}}
		{\footnotesize\hspace{.4mm}\nms^{\mkern-13mu\tiny\uuparrow#1}}
		{\scriptsize\nms^{\mkern-10mu\tiny\tuuparrow#1}}
	}
}

\newcommand{\mqmc}{\mathrel{
		\mathchoice
		{\hspace{.4mm}\nms^{\mkern-18mu\scriptsize\uuparrow}\hspace{.6mm}}
		{\hspace{.4mm}\nms^{\mkern-18mu\scriptsize\uuparrow}\hspace{.6mm}}
		{\footnotesize\hspace{.4mm}\nms^{\mkern-11mu\tiny\uuparrow}\hspace{.6mm}}
		{\scriptsize\nms^{\mkern-10mu\tiny\tuuparrow}}
	}
}

\newcommand{\mrc}[1]{\mathbin{
		\mathchoice
		{\normalsize\hspace{.5mm}\nms^{\mkern-19mu\scriptsize\Uuparrow#1}\hspace{-.5mm}}
		{\normalsize\hspace{.5mm}\nms^{\mkern-19mu\scriptsize\Uuparrow#1}\hspace{-.5mm}}
		{\footnotesize\hspace{.5mm}\nms^{\mkern-13.5mu\fontsize{5.5}{0}\Uuparrow#1}}
		{\scriptsize\nms^{\mkern-10mu\tiny\Uuparrow#1}}
	}
}

\newcommand{\smc}{\mathbin{
		\mathchoice
		{\hspace{.4mm}\nms^{\mkern-17mu\scriptsize\thuarrow}\hspace{.6mm}}
		{\hspace{.4mm}\nms^{\mkern-17mu\scriptsize\thuarrow}\hspace{.6mm}}
		{\footnotesize\hspace{.4mm}\nms^{\mkern-11mu\tiny\thuarrow}\hspace{.6mm}}
		{\scriptsize\nms^{\mkern-10mu\tiny\thuarrow}}
	}
}
\newcommand{\nnmc}{\not\nmc}
\newcommand{\nsmc}{\not\mkern-3mu\smc}
\newcommand{\nmrc}{\not\mkern-3mu\mrc}
\newcommand{\nmqmc}{\not\mkern1mu\mqmc}
\newcommand{\nqmc}{\not\mkern1mu\qmc}

\newcommand{\nme}{%
	\mathbin{\mathpalette\@nme\expandafter}
}
\newcommand{\@nme}{{\mid\joinrel\mkern-.5mu\sim\mkern-2mu}_{e}\mkern3mu}


%\newcommand{\nme}{\nms\mkern-7mu_{e}\mkern2mu}
\newcommand{\qme}{{\qmc\mkern-2mu}_{e}\mkern3mu}
\newcommand{\mqme}{{\mqme\mkern-2mu}_{e}\mkern3mu}
\newcommand{\sme}{{\smc\mkern-2mu}_{e}\mkern3mu}

% % % % % % % % % % %Commands for Materia Incoherence% % % % % % % % % % % %

\newcommand{\bigperpp}{%
	\mathop{\mathpalette\bigp@rpp\relax}%
	\displaylimits
}
\newcommand{\bigp@rpp}[2]{%
	\vcenter{
		\m@th\hbox{\scalebox{\ifx#1\displaystyle1.3\else1.3\fi}{$#1\perp$}}
	}%
}
\newcommand{\bigperp}{\raisemath{.5pt}{\bigperpp}}

%% % % % % % Degree Command % % % % % % % % % % % % % % % % % % % %
\newcommand{\degree}{\ensuremath{^\circ}}

%%%%%%%%%%%%%Author Comments%%%%%%%%%%%%%%%%%
 \newcommand{\kk}[1]{\textcolor{red}{$^{\textrm{KK}}${#1}}}
 \newcommand{\jm}[1]{\textcolor{blue}{$^{\textrm{JM}}${#1}}}
 \newcommand{\mr}[1]{\textcolor{green}{$^{\textrm{MR}}${#1}}}


\makeatother





%\renewcommand{\@cite}[1]{#1}
%\usepackage[natbibapa]{apacite}
\usepackage[hang,flushmargin]{footmisc} 
\usepackage[hidelinks]{hyperref}
%\usepackage{lingmacros}
%\hypersetup{
%    colorlinks=false,
%    pdfborder={0 0 0},
%}


\begin{document}
\sloppy
\title{Explanatory Asymmetry and Inferential Practice
%\thanks{Grants or other notes
%about the article that should go on the front page should be
%placed here. General acknowledgments should be placed at the end of the article.}
}


\raggedbottom

\maketitle



\section{Material Consequence Relations}\label{sec:MatCon}

In this appendix, we provide a formal account of the conditions necessary for a consequence relation to qualify as \textit{sturdy}. Since sturdy inferences are a species of material inferences, we begin with a material consequence relation defined over a finite language, $ \mathcal{L}_{0} $, consisting of atomic sentences $ p, p_1, p_2, \ldots, p_n$. Let $ A, B, C, D $ range over sentences; $ \Gamma, \Delta$ over sets of sentences; and $ W, W',W'' $ over sets of sets of sentences. In order to represent material incoherence, we extend $ \mathcal{L}_0 $ to include the constant $ \bigperp $, i.e. $ \mathcal{L} $ = $ \mathcal{L}_0 \,\,\bigcup \,\,\{\bigperp\}$. We stipulate that $ \bigperp $ can neither appear to the left of the turnstile nor be embedded. Thus, $\nmc$ maps sets of sets of sentences in $ \mathcal{L}_{0} $ to sentences in $ \mathcal{L}$, i.e. $\nmc\subseteq \mathcal{P}(\mathcal{L}_0) \times \mathcal{L} $. We define $ \nmc $ as the relation over $ \mathcal{L} $ such that $  \Gamma\nmc A $ iff  $\Gamma$ materially implies $ A $, and $\Gamma\nmc\!\!\bigperp $ iff $ \Gamma $ is materially incoherent. In keeping with our goal of articulating the inferential patterns that underwrite our ``exhaustive explanations,'' we will on occasion abuse our notation and use ``$\ushort{A}$'' to denote the set $ \{A_1, \dots A_n\} $. Paradigmatically, we will write $\Gamma,\, \ushort{A} \nmc B $ for $\Gamma, A_1,\ldots,A_n \nmc B $. 

We now have in place an object language $ \mathcal{L} $ (an extension of  $ \mathcal{L}_0 $ that includes $ \bigperp $) and a meta-language that consists, \textit{inter alia}, of $ \nmc $.  We use a meta-theoretical conditional, $ \Longrightarrow $, to formulate the properties of the structure $ \langle \mathcal{L}, \nmc \rangle $ as follows:

\begin{enumerate}
	\item $ \mathcal{L}_0\nmc\bigperp$
	\item $ \emptyset\nnmc\bigperp $
	\item $ \forall p, \forall\Delta\subseteq\mathcal{L}_0(\Gamma, \Delta \nmc\bigperp \Longrightarrow \Gamma\nmc p) $ (Ex Falso Fixo Quodlibet)
	\item $\forall \Gamma\subseteq\mathcal{L}_0 (\Delta \in \Gamma \Longrightarrow \Gamma\nmc \Delta)$ (Reflexivity)
	\item $\forall\Delta\subseteq\mathcal{L}_0(\Gamma\nmc p \,\,\not\!\!\Longrightarrow \Gamma, \Delta \nmc p)$ (Nonmonotonicity)
	\item $(\Gamma\nmc p_j $ and $ \Gamma, p_j \nmc p_k ) \,\,\not\!\!\Longrightarrow \Gamma \nmc p_k$ (Non-transitivity)
\end{enumerate}

The first two properties state, respectively, that the totality of $ \mathcal{L} $ is incoherent and that the empty set is not. The principle of \textit{Ex Falso Fixo Quodlibet} is a modification of \textit{Ex Falso Quodlibet} that restricts `explosion' to monotonic contexts only. The rationale for this restriction is that if a set of atomic sentences is nonmonotonically incoherent, then adding additional sentences to it may make it coherent, and therefore we are not licensed to infer an arbitrary atom from it (the original set). Reflexivity is a standard property of consequence relations and nonmonotonicity has been explained above. 

The last property of the structure, however, is sure to surprise the reader. Nearly all consequence relations, including the standard nonmonotonic ones, are transitive and include the Cut rule as a structural rule. Nevertheless, as \textcite{Morgan2000} has demonstrated,  a deduction theorem (i.e. $\Gamma, A \nmc B \Longleftrightarrow \Gamma \nmc A \rightarrow B $) cannot hold for a nonmonotonic, reflexive, and transitive consequence relation.  The provability of a deduction theorem is essential to our inferentialist approach to logical vocabulary, since it establishes that an object-language operator gives expression to, i.e. makes explicit, what would be given as a rule of inference in the meta-language, i.e. using `$ \nmc $'. We have already motivated the nonmonotonic character of material inference above. So, we must choose between  reflexivity and transitivity. 

There are some reasons to prefer a consequence relation that is reflexive but non-transitive.  First, logicians of inferentialist and proof-theoretic persuasion have already explored systems in which Cut fails \citep{Ripley2011,Tennant2014}, whereas none seems inclined to pursue non-reflexive ones. Second, Ulf \textcite{Hlobil2016}  defines a sequent calculus over structures with roughly the same properties as  $ \langle \mathcal{L}, \nmc \rangle $ that includes left- and right-rules for several logical operators and proves that they are conservative extensions of an atomic language like our $ \mathcal{L}_0 $. In subsequent work, we hope to build on this system and to develop calculi with rules for logical and explanatory vocabulary. Finally, sets of explanatory statements often fail to license transitive inferences---e.g. the occurrence of the Big Bang does not explain why Adam ate the apple, even if there are true explanatory statements linking the Big Bang to event $E_1$, $ E_1 $ with $ E_2 $, and so on up to Adam's eating of the apple. Of course, reflexivity is not a property of explanations either, but we can capture this fact with appropriate constraints on sturdy inferences.

As noted above, sturdy inferences belong to the class of modally robust inferences---i.e. those material consequence relations that \textit{would} remain good under some range of suppositions consistent with the premises. In order to represent the range of modal robustness exhibited by nonmonotonic material consequence relations we first quantify over $ \nmc $.\\

\noindent\textbf{Quantified Material Consequence (QMC):}\label{QMC}
\begin{equation}
\Gamma, A \qmc{W} B \Longleftrightarrow_{df}
\begin{cases}\nonumber
1.\,\, W\subseteq\mathcal{P}(\mathcal{L}) & $ and $ \\[3pt] 
2.\,\, \forall\Delta\in W(\Gamma, A,  \Delta \nmc B ) 
\end{cases}
\end{equation}

\hyperref[QMC]{\textbf{QMC}} provides a precise way of talking about \textit{sets} of material inferences. Since $ \emptyset \in W $, our set must include $ \Gamma, A \nmc B $. Call this the \textit{base inference} of the set. Sets of inferences are delimited by the set of sets of sentences of $ \mathcal{L} $ whose members may be added to the premises without impugning the base inference. The resulting set thus contains $ |W| $ many inferences. When $W$ is the set of all sets of the language, i.e. $ \mathcal{P}(\mathcal{L}) $, the consequence relation is globally monotonic---in which case we simply write $ \smash{\mqmc}$. 

The reason for circumscribing these sets of inferences by subsets of $ \mathcal{P}(\mathcal{L}) $ is that atomic sentences that would, by themselves materially defeat an inference if added to the premises, need not defeat that inference if they are added along with other atomic sentences. As noted above, we think of these sets of atomic sentences as \textit{suppositions} and of $W$ as a set of such suppositions. Thus $ \smash{\Gamma, A \qmc{W} B} $ says that the (base) inference from $ A $ to $ B $ \textit{would} remain materially good in context $ \Gamma $, even if the suppositions in $ W $  \textit{were} to hold. 

However, $ \smash{\Gamma, A \qmc{W} B} $ does not imply that the base inference would become materially bad if we were to add something from \textit{outside} of $W$, and so \hyperref[QMC]{\textbf{QMC}} does not capture an inference's range of modal robustness. This range consists in the set of \textit{all} suppositions that could be added to its premises without undermining its correctness. The following definition permits us to identify consequence relations by their range of modal robustness. (For the remaining meta-theoretical claims, we omit reference to $\mathcal{L}$ and $\mathcal{P}(\mathcal{L})$.)\\

\noindent\textbf{Modally Robust Consequence (MRC):}\label{MRC}
\begin{equation}
\Gamma, A \mrc{W} B \Longleftrightarrow_{df}
\begin{cases}\nonumber
1.\,\, \Gamma, A \qmc{W} B& $ and $ \\[3pt] 
2.\,\, \forall W'(\Gamma, A \qmc{W'} B \Longrightarrow W' \subseteq W)  & $ and $\\[3pt] 
%3.\,\, \nqmc[]{\Gamma}{B}{W} & $ and $\\[3pt]  
3.\,\, \forall\Delta(\Gamma, A, \Delta \mqmc \bigperp \Longrightarrow \Delta \not\in W)

\end{cases}
\end{equation}

The second condition of \hyperref[MRC]{\textbf{MRC}} says that $W$ contains \textit{all} those sets of sentences whose addition would not defeat the base inference. The third condition prohibits the elements of $W$ from being logically inconsistent with the premises, thereby ensuring that the conclusion does not follow trivially from \textit{Ex Falso Fixo Quodlibet}. When the consequence relation $ \smash{\mrc{W}} $ holds, we say that the base inference is \textit{modally robust up to} $ W $.

A sturdy inference must not only be modally robust; it must be superlatively stable. In other words, there must be no material inference with the same conclusion in the same context whose range of modal robustness is strictly greater than that of a sturdy inference. The following meta-theoretical conditional, in which $\smash{\smc}$ denotes sturdy consequence relations, captures these characteristics.\\ 

\noindent\textbf{Sturdy  Consequence (SC):}\label{SC}
\begin{equation}
\Gamma, \,A \smc B \Longrightarrow
\begin{cases}\nonumber 
1.\,\, \Gamma, \, A \mrc{W} B & $ where $ A \neq B $, and $ \\[3pt] 
2.\,\, \forall C\,(\Gamma, C \mrc{W'} B \Longrightarrow W \not\subset W' ) & $ where $ B\neq C  \\[3pt] 
\end{cases}
\end{equation}

The first condition states that sturdy inferences are modally robust, with the proviso that their non-contextual premises are distinct from their conclusions. This ensures that unlike other material inferences, sturdy inferences are not reflexive. The second condition of \hyperref[SC]{\textbf{SC}} says that a sturdy inference must be modally robust up to some $W$ such that $W$ is not included in any other set of suppositions under which any sentence from $\mathcal{L}$ materially implies $B$, excluding $B$ itself. Since a sturdy inference's island of monotonicity is not encompassed by any other island from which the conclusion can be drawn, there is no further need to compare sets of suppositions, and thus the turnstile for the sturdy consequence relation may omit `$W$'.


\section{Sequent Calculus for Sturdy Inferences}\label{sec:SeqCal}
We show how this structure can be extended to one with a language $ \mathcal{L}_e $ that contains an expression for \textit{best explains}, namely `$ \twoheadrightarrow $' and construct a consequence relation, `$ \nme $' by means of a sequent calculus that takes material implications in $ \mathcal{L} $ as axioms. We call this calculus \textbf{EL} for ``Explanatory Logic''. Let $ \mathcal{L}_0^\twoheadrightarrow = \mathcal{L}_0 \cup \{\twoheadrightarrow\} $ and $ \mathcal{L}_e = \mathcal{L}_0^\twoheadrightarrow \cup \{\bigperp\} $. We read the formula $ \ushort{A} \twoheadrightarrow  B $ as ``$ \ushort{A} $ (exhaustively) explains $ B $'' or as ``$ \ushort{A} $ is an (exhaustive) explanation of $ B $''. We will now give the axioms and rules for the sequent calculus \textbf{EL} that extends the material consequence relation and its explanatory variant to  $ \nme \subseteq \mathcal{P}(\mathcal{L}_0^\twoheadrightarrow) \times \mathcal{L}_e$.

\begin{description}
	
	\item[\underline{Axiom 1}] If $ \Gamma, \ushort{A} \nmc B  $ then $ \Gamma, \ushort{A} \nme B  $ is an axiom.
	\item[\underline{Axiom 2}] If $ \Gamma, \ushort{A} \smc B $ then $ \Gamma, \ushort{A} \sme B$ is an axiom.
\end{description}

The rules for \textbf{EL} are as follows:

\begin{prooftree}
\def\fCenter{\ \nme\ }
\AxiomC{$ \Gamma,\, \ushort{A} \sme B $}
\RightLabel{\hspace{5mm}  EER}
\UnaryInf$\Gamma \fCenter \ushort{A} \mathbin{\twoheadrightarrow}B$
\end{prooftree}

\begin{prooftree}
\def\fCenter{\ \nme \ }
\AxiomC{$\Gamma \fCenter \ushort{A} \twoheadrightarrow B$}
\RightLabel{\hspace{5mm}  EES}
\UnaryInf$\Gamma, \,\ushort{A} \fCenter B $
\end{prooftree}

\begin{prooftree}
\def\fCenter{\ \nme\ }
\AxiomC{$ \Gamma, \,\ushort{A} \sme B$}
\AxiomC{$\Gamma \fCenter \hspace{-1ex}  B $}
\RightLabel{\hspace{5mm}  EEL}
\BinaryInf$\Gamma, B, \ushort{A} \twoheadrightarrow B\fCenter \hspace{-1ex}  A$
\end{prooftree}


\printbibliography


\end{document}
