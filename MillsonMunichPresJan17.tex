\documentclass[xcolor=dvipsnames,12pt,handout]{beamer} 
\definecolor{lightgray}{gray}{.8}
\definecolor{ultralightgray}{gray}{.95}
\mode<presentation>
{
  \usetheme{Goettingen}      % or try Darmstadt, Madrid, Warsaw, ...
    \usecolortheme{beaver} % or try albatross, beaver, crane, ...
    \usefonttheme{structurebold}  % or try serif, structurebold, ...
    \setbeamertemplate{navigation symbols}{}
    \setbeamertemplate{caption}[numbered]
    \setbeamertemplate{items}[ball] 
    \setbeamertemplate{blocks}[rounded][shadow=true]
    \setbeamercovered{transparent}
    \setbeamercolor{block body}{bg=ultralightgray}
    \setbeamercolor{block title}{bg=lightgray}
    \setbeamercolor{title in sidebar}{fg=black}
    \setbeamercolor{section in sidebar shaded}{fg=brown}
    \setbeamercolor{section in sidebar}{bg=brown}
    }

\usepackage[english]{babel}
\usepackage[utf8x]{inputenc}
\usepackage{graphicx}
\usepackage{latexsym}
\usepackage{amsthm}
\usepackage{amsmath}
\usepackage{turnstile}
\usepackage{amsthm}
\usepackage{mathrsfs}
\usepackage{ebproof}
\usepackage{times}


%\usepackage[autostyle]{csquotes}
%\usepackage[doi=false,isbn=false,url=false,style=chicago-authordate,natbib=true]{biblatex}
%
%\addbibresource{KMR_Master.bib}

\usepackage{setspace}% http://ctan.org/pkg/setspace
\let\oldframetitle\frametitle% Store \frametitle in \oldframetitle
\renewcommand{\frametitle}[1]{%
  \oldframetitle{#1}\setstretch{1.2}}


\title{Explanatory Asymmetry and Inferentialist Expressivism}
\author{ Kareem Khalifa $ \cdot $  Jared Millson$ \cdot $ Mark Risjord }
\date{January 2017}
\institute{Jared Millson\\ Department of Philosophy, Agnes Scott College\\[1ex]
  \texttt{jmillson@agnesscott.edu}}
\begin{document}

\begin{frame}
\titlepage
\end{frame}

\section{Inference to the Best Explanation}


\begin{frame}\frametitle{Issues with IBE}
	\begin{block}{IBE Argument Form}	
		\begin{itemize}
			\item[] $P$ is the best explanation of $Q$
			\item[] $Q$
			\item[] Therefore, $P$
		\end{itemize}
	\end{block}
\pause
	\vspace{.5cm}
	Philosophical issues
	\begin{itemize}
		\item What is the status of IBE as a form of inference?
			\begin{itemize}
				\item Is IBE reducible to other forms of inference?
				\item Are other forms of inference reducible to IBE?
			\end{itemize}
		\item<3> What is ``explanation'' and what makes one better than another?
	\end{itemize}	
\end{frame}

\section{Theories of Explanation}

\begin{frame}\frametitle{Theories of Explanation}
	
	Three kinds of philosophical account of explanation
	\begin{itemize}
		\item Inferential (Hempel, Kitcher)
		\item Causal  (Lewis, Salmon, Lipton)
		\item Pragmatic  (Van Fraassen, Achinstein, Risjord)
	\end{itemize}
\pause
	None are adequate as a basis for IBE
	\begin{itemize} 
		\item Too narrow, and plausible IBEs must be gerrymandered to fit
		\item Too broad, and non-IBEs must be accepted
	\end{itemize}
\end{frame}


\begin{frame}{Disclaimer}
\begin{block}{Warning!}
Throughout this presentation  you will see and hear highly offensive and derogatory words.
%\begin{itemize}
%\item Statuses: commitment and entitlement
%\item Attitudes: undertaking and attributing 
%\end{itemize}
\end{block}
\pause
But rest assured that these words will only be mentioned and will never be used!
\pause
\begin{block}{The Use/Mention Distinction}
\pause
\begin{enumerate}
\item Atlanta has great restaurants.
\pause
\item Atlanta has seven letters.
\pause
\item `Atlanta' has seven letters.
\end{enumerate}
\pause
...and each of these sentences is mentioned (as an example), not used.
\end{block}
\end{frame}

\begin{frame}{Disclaimer}
\vspace{-1cm}
\begin{itemize}\setlength\itemsep{2.5mm}
\item So, slurs that are merely mentioned won't have derogatory force, right?
\pause
\item Unfortunately...the offensive power of slurs tends not to respect the use/mention distinction.
\pause
\begin{itemize}\setlength\itemsep{2.5mm}
\item The mere pronunciation of a slur can be offensive.
\pause
\item Especially when the speaker occupies a position of social privilege
\pause
\item ...though slurs mentioned in writing are less prone to offend than those mentioned out loud.
\end{itemize}
\end{itemize}
\end{frame}



\begin{frame}{Euphemisms}


\vspace{-1cm}
Why not use euphemisms like 'the N-word'?
\pause
\begin{itemize}\setlength\itemsep{2mm}
\item Philosophers of language use example sentences to pump intuitions about the meaning, meaningfulness, appropriateness, expressive force, etc.
\pause
\item But, euphemisms are designed to block (at least some of) the intuitive reactions we have to the use of offensive expressions.
\pause
\item So, example sentences that substitute euphemisms for slurs would not generate useful data.
\end{itemize}  


\end{frame}

\begin{frame}{Policy for the talk}


\vspace{-1cm}

\pause
\begin{itemize}\setlength\itemsep{6mm}
\item Example sentences containing highly offensive slurs will only be written 
\pause
\item Only nationality-based slurs will be pronounced, since they tend to be less offensive.
\pause
\item Beg for benefit of the doubt! We're trying to understand what makes these words so harmful.
\end{itemize}  


\end{frame}



\begin{frame}{What is a slur?}

\vspace{-.5cm}
\begin{block}{Working Definition}
{\small Slurs are expressions that are typically used}
\pause
	\begin{enumerate}
	{\small \item to insult, denigrate, or otherwise `put down' individuals on the basis of their membership in a group or 
	\pause
	\item to refer to a group or one of its members in a derogatory, contemptuous, or insulting manner.}
	\end{enumerate}
\end{block}

\pause
Slurs are a type of pejorative language, which also includes:
\pause
\begin{block}{Pejoratives}
\begin{itemize}
\item Expletives (e.g. `Damn!' `Shit!' `Fuck!')
\pause
\item Vulgarities (e.g. `cock', `ass,' `fuck,' `shit')
\pause
\item Insults (e.g. `jerk,' `asshole,' `douche bag')
\end{itemize}
\end{block}

\end{frame}

\begin{frame}{What is a slur?}

\begin{block}{Working Definition}
Slurs are expressions that are typically used
	\begin{enumerate}
	\item to insult, denigrate, or otherwise `put down' individuals \alert<2->{on the basis of their membership in a group} or 
	\item to refer to \alert<2->{a group or one of its members} in a derogatory, contemptuous, or insulting manner.
	\end{enumerate}
\end{block}
\pause
\pause
\begin{block}{Types of Slurs}
\begin{itemize}
\item Ethinic/Racial: (e.g. `nigger' `kike' `spic')
\pause
\item Gender (e.g. `pricl,' `bitch,' `cunt')
\pause
\item Sexual Orientation (e.g. `faggot,' `dyke,' `queer')
\pause
\item Nationality (e.g. `chink,' `limey,' `frog,' `kraut')
\end{itemize}
\end{block}

\end{frame}


\section{What's so interesting?}
\begin{frame}{The Linguistic Behavior of Slurs}

\begin{itemize}
\item \textit{Attitude Expression}: use of a slur is typically taken to indicate that the speaker holds derogatory attitudes.
\pause
\medskip
\item \textit{Derogatory Autonomy}: slurs are often derogatory even when the speaker does not intend the use to be derogatory.
\begin{itemize}
\item `I love spics; they are my favorite people on Earth'
\end{itemize}
\medskip
\pause
\item \textit{Bystander Complicity}: in some cases, merely overhearing a slur is sufficient for making a non-prejudiced listener feel complicit in a speaker’s slurring performance.
\medskip
\pause
\item \textit{Offensive Variation}: not all slurs, even if co-referential, appear to be equally offensive. 
\begin{itemize}
\item `colored' v.s. `nigger'
\end{itemize}


\end{itemize}
\end{frame}

\begin{frame}{The Linguistic Behavior of Slurs}
\begin{itemize}
\vspace{-.5cm}
\pause
\item \textit{Appropriation}: the target group can sometime `take over' a slur and transform its meaning to lessen or to eliminate its derogatory force.
\bigskip
\pause
\item \textit{Embedding Failure}: the derogative force and offensiveness of slurs projects out of various forms of embedding, including
indirect reports, negations, conditionals, and mentions.
\pause
\begin{enumerate}
\item Susan said that the town is full of dykes.
\item Jim is not a spic.
\item If I hated niggers, I'd be racist.
\end{enumerate}
\bigskip
\pause
\item \textit{Insulation}: slurs can be used inoffensively, e.g. official anti-discrimination policies, among bigots, comedy.

\end{itemize}
\end{frame}

\begin{frame}{Desiderata of a Theory}
\begin{itemize}
\vspace{-1cm}
\item We want a simple, consistent theory that explains/predicts as much as of the behavior of slurs as possible.
\bigskip
\pause
\item We want a theory that coheres with our best theories of linguistic meaning (semantics) and of language-use (pragmatics).
\bigskip
\pause
\item It'd be great if it had clear implications for how we ought to treat the appearance of slurs in speech and writing.

\end{itemize}
\end{frame}

\begin{frame}{Semantics or Pragmatics}
\begin{block}{Two directions for a theory of slurs}
\begin{enumerate}
\pause
\item \textit{Semantic:} Most of the behavior of slurs can be explained by their content, \textcolor{blue}{i.e. what is literally said or conventionally implied when a slur is uttered}.
\bigskip
\pause
\item \textit{Pragmatic:} Most of the behavior of slurs can be explained by facts about \textcolor{blue}{who uses or has used them, what the speaker's intention is, what the context is like, what the typical effects on an audience are, what social institutions are in play etc.}.
\end{enumerate}
\end{block}
\end{frame}

\section{Some Theories of Slurs}
\subsection{Semantic Approaches}
\begin{frame}{Neutral Counterparts}
\begin{block}{Semantic Assumptions}
\begin{itemize}
\pause
\item \textbf{The \textit{Weak} Neutral Counterpart Assumption}: All or most slurs have non-derogatory counterparts that \textit{co-refer}.
\pause
\begin{itemize} 
\item `Batman' and `Bruce Wayne' co-fer, just as
\pause
\item `Kraut' and `German'
\pause
\item `Nigger' and `African-American'
\pause
\item `Kike' and `Jew'
\end{itemize}
\medskip
\pause
\item \textbf{The \textit{Strong} Neutral Counterpart Assumption}: All or most slurs have non-derogatory counterparts, with which they are \textit{synonymous} (e.g. `bachelor' and `unmarried male').
\end{itemize}
\end{block}
\end{frame}

%\begin{frame}{Pure Expressivism}
%\begin{block}{Pure Expressivism}
% Slurs do not serve to describe; instead they express the speaker's emotional state (e.g. disapproval). 
%\end{block}
%\begin{block}{Non-displacement}
%\pause
%Unlike pure descriptives (e.g. `farmer') pure expressives (e.g. `fucker') can't be (felicitously) combined with denials of emotional expression. The emotional expression can't be `displaced.'
%\pause
%\begin{enumerate} 
%{\small \item I met a farmer, but I deny expressing any feelings about him.
%\pause
%\item I met a fucker, but I deny expressing any feelings about him.*
%\pause
%\item I met a kraut, but I deny expressing any feelings about him.*}
%\end{enumerate}
%\end{block}
%\end{frame}

\begin{frame}{Pure Expressivism}
\pause
\begin{block}{Pure Expressivism}
 Slurs just express the speaker's non-cognitive state (e.g. disapproval), like the pejoratives `the damn dog' or `that fucking cat'.
 \begin{itemize}
 \item Sentences with slurs (e.g. Sam is a limey) are neither true nor false. 
 \end{itemize} 
\end{block}
\begin{block}{Classificatory Error}
\pause
If slurs are pure expressives, then speakers never commit classificatory errors in using them. And yet...
\pause
\begin{itemize} 
\item[A] Jim is a kraut.
\pause
\item[B] No he isn't!
\pause
\item  No he isn't! = Jim is not German. \alert{(descriptive)}
\pause
\item[B$'$] No he isn't! He's Italian.
\end{itemize}
\end{block}
\pause
\medskip
So, it looks as though slurs do have a descriptive content.
\end{frame}

\begin{frame}{Hybrid Expressivism}
\begin{block}{Hybrid Expressivism}
 Slurs contain both expressive and descriptive content.
 \pause
 \begin{itemize}
 \item 'Jim is a kraut' expresses both
 \begin{enumerate}
 \item the pure belief that Jim is German
 \item a (negative) cognitive-affective state toward Germans (`Germans (Boo!); they deserve contempt')
 \end{enumerate}
 \pause
 \item Assumes Weak Neutral Counterparts.
 \end{itemize}
\end{block}

\end{frame}

\begin{frame}{Problems with Hybrid Expressivism}
\begin{block}{Explantory Power of HE}
\pause
\begin{itemize}\setlength\itemsep{2.5mm}
 
\item Explains classificatory error, attitude expression, and some embedding failure (conditionals \& negations).
\pause
\item Trouble explaining:
\pause
\begin{itemize}\setlength\itemsep{2.5mm}
\item derogatory autonomy
\pause
\item offensive variation (are attitudes that fine-grained?)
\pause
\item appropriation (derogation is part of literal meaning)
\pause
\item insulation (use of the slur still expresses insulting attitude)
\end{itemize} 

\end{itemize}
\end{block}
\end{frame}


\begin{frame}{Combinatorial Externalism (CE)}
\pause
\begin{block}{CE for Slurs}
\begin{itemize}
\item the use of a slur expresses a \alert{combination} of evaluative and descriptive concepts, similar to `cruel' or `courageous.'
\pause
\item what is expressed in the use of a slur is in part determined by \alert{external} factors, (i.e. `not in your head'), namely, by the social institutions that discriminate on the basis of negative stereotypes.
\end{itemize}
\end{block}
\end{frame}

\begin{frame}{Combinatorial Externalism (CE)}

\begin{block}{CE Analysis of Slurs}
\pause
\begin{itemize}
\item S = a slur
\pause
\item N = the neutral counterpart to S
\pause 
\item `X is an S' means...
\pause
\item \textit{X ought to be subject to such-and-such discriminatory practices because of having such-and-such negative properties, all because of being N.}
\pause
\item The discriminatory practices and negative properties are fixed by external factors: racist/sexist ideology and institutions.
\end{itemize}
\end{block}

\end{frame}

\begin{frame}{Combinatorial Externalism (CE)}
\pause
\begin{block}{CE Analysis of `Chink' (Hom and May, 2008)}
\begin{enumerate}
\item[a.] `Yao is a chink' is analyzed as...
\pause
\item[a$'$.] Yao ought to be subject to higher college admissions standards, and exclusion from advancement to managerial positions, etc., because of being slanty-eyed, and devious, and good-at-laundering, etc., all because of being Chinese.
\end{enumerate}
\end{block}
\end{frame}

\begin{frame}{Combinatorial Externalism (CE)}

\begin{itemize}
\vspace{-.4cm}
\item[a.] `Yao is a chink'
\end{itemize}

\begin{block}{Motivation}
Hom and May think that
\begin{itemize}
\item Non-racists can and should respond to sentences like \textcolor{blue}{a.} by saying:
\pause
\begin{itemize}
\item[b.] `There are no chinks; there are only Chinese'
\pause
\end{itemize}
\item Moral Fact:\textit{no one ought to be subject to discriminatory practices simply because of their membership in a group}.
\pause
\item So, \textcolor{blue}{a.} is false, and \textcolor{blue}{b.} is true.
\pause
\item CE + Moral Fact explains why slurs have \textit{null extensions}
\pause
\item CE denies Weak Neutral Counterpart assumption

\end{itemize}
\end{block}

\end{frame}



\begin{frame}{Combinatorial Externalism (CE)}

\begin{block}{Explanatory Power of CE}
\pause
\begin{itemize}
\item The externalist feature of CE accounts for derogatory autonomy and, perhaps, offensive variation.
\pause
\medskip
\begin{itemize}
\item Social institutions determine the meaning of slurs, not the attitudes of individuals.
\end{itemize}
\pause
\medskip
\item CE accounts for appropriation by in-group members.
\pause
\begin{itemize}
\item Reclaimers create a new supporting social institution for the term which imbues it with new meaning.
\end{itemize}
\pause
\item CE can explain attitude expression: Speakers are taken to have a derogatory belief when they use a slur.
\pause
\item Insulation is by-passed: offensiveness is a function of contexts on utterances.
\end{itemize}
\end{block}
\end{frame}



\begin{frame}{Combinatorial Externalism (CE)}

\begin{block}{Against CE}
\begin{itemize}
\item CE produces the wrong analysis for slurs that ae embedded, e.g. in negations.
\pause
\begin{itemize}
\item[b.] `There will never be a chink President.'
\pause
\item[b$'$.] \textit{There will never be a President who ought to be excluded from advancement to managerial positions, etc., because of being devious, etc., all because of being Chinese.}
\pause
\item But b$'$ it is not racist; in fact, it seems to express admirably nonracist sentiments.
\end{itemize}
\pause
\item Moreover, under CE analysis, \textcolor{blue}{d.} fails to predict that the speaker is denying that Paul is Jewish.
\item[d.] `Paul's not a kike. He's a kraut.'
\pause
\end{itemize}
\end{block}

\end{frame}

\subsection{Pragmatic Approaches}
\begin{frame}{Pragmatic Approaches}
\vspace{-.4cm}
\pause
\begin{itemize}
\item Semantic accounts of slurs assume that bigots use slurs because their literal meaning is derogatory.
\pause
\item But what if it's the other way around.
\pause
\item What if slurs are derogatory because racists, misogynists, and homophobes, etc. use them?
\end{itemize}
\pause
\begin{block}{The meaning of slurs isn't special}
\begin{itemize}
\item The bulk of their offensive behavior is due to pragmatics (who uses them, when, where, how, with whom,etc.).
\pause
\item The semantic content of slurs is just that of their neutral counterparts 
\pause
\item Strong Neutral Counterpart Assumption
\end{itemize}
\end{block}


\end{frame}

\begin{frame}{Prohibitionism}

\begin{block}{Slurs are prohibited words}
\begin{itemize}
\pause
\item It is not because of what one's use of a slur is expressive of that one offends.
\pause
\item Rather, it is because one uses a word which (for certain reasons) is taboo.
\pause
\item Slurs are offensive because it violates a prohibition
\end{itemize}
\end{block}


\begin{itemize}

\pause
\item Once relevant individuals declare a word a slur, it becomes one.
\pause
\item Prohibition can occur for a number of reasons
\begin{enumerate}
\item their use is associated with acts of denigration and insult or those who hold attitudes of contempt
\pause
\item deprives target group of authority to decide their name
\pause
\item reminder of oppression
\end{enumerate}
\end{itemize}

\end{frame}

\begin{frame}{Prohibitionism}

\begin{block}{Explanatory Power}
Prohibition explains
\begin{itemize}
\pause
\item derogatory autonomy (doesn't matter what your attitude is)
\pause
\item offensive variation (sanctions for violating taboo come in degree)	
\pause
\item embedding failures (the prohibition is on the very appearance of the word)
\pause
\item bystander complicity (we are all responsible for enforcing the prohibition)
\pause
\item appropriation (members of target group use an 'escape clause' in the prohibition)
\end{itemize}
\end{block}


\end{frame}

\begin{frame}{Prohibitionism}

Some problems with Prohibitionism

\begin{itemize}
\pause
\item Can't a word be a slur in the absence of taboo?
\pause
\item Taboo-breaking captures no difference in offense to targets and non-targets
\pause
\item bystander complicity:  I don't feel at risk of being implicated in the bad behavior of some who uses a swear word.	
\pause
\item swear words are prohibited in polite contexts, but not in others.
\pause
\item Isn't it likely that one of the reasons slurs are prohibited is because their meaning is derogatory?
\end{itemize}

\end{frame}

\section{Policy Implications}

\begin{frame}{What to do about slurs}

If we endorse one of these theories, and assuming that we want to mitigate derogation and offense what actions can or should we adopt toward the appearance/use of slurs?
\begin{block}{Expressivism}
\pause
Policy: impose/maintain prohibitions on the utterance of slurs across all contexts. 
\pause
	\begin{itemize}
	\item Appropriation of slurs by target groups introduces ambiguity and target groups are in no privileged position for \textit{reclamation} v.s. melioration.
	\pause
	\item So its possible, but risky.   
	\end{itemize}

\end{block}

\end{frame}

\begin{frame}{What to do about slurs}

\vspace{-1cm}
\begin{block}{Combinatorial Externalism}

\pause
Policy: dismantle institutions and practices that perpetuate racism and morally (re)educate racists.
\pause
	\begin{itemize}
	\item Reclamation of slurs by target groups is again risky, but achievable by erecting counter-practices.
	\end{itemize}

\end{block}
\end{frame}

\begin{frame}{What to do about slurs}
\vspace{-1cm}

\begin{block}{Prohibitionism}
\pause

Policy (according to proponents): many slurs should be eradicated entirely from common usage, even among in-group speakers.

\pause
	\begin{itemize}
	\item What about the Lenny-Bruce-Strategy?
	\pause
	\item Lenny Bruce: ``the suppression of \textit{the word} gives it the power, the violence, the viciousness.''
	\pause
	\item So radically breaking the taboo through over-use, we deprive the words of their offensive power.
	\end{itemize}

\end{block} 
\end{frame}


\begin{frame}
\begin{block}{}
\vspace{3cm}
\centering{\Huge THANK YOU!}
\vspace{3cm}
\end{block} 
\end{frame}



%\begin{frame}{The Contrastive Choice Account}
%
%\pause
%Taking a clue from the pragmatics of impolite behavior...
%
%\begin{itemize}
%\item Expressions get associated with certain attitudes when they are regularly chosen over other semantically equivalent ones.
%\pause
%
%\item An impolite utterance generates offense in a context when
%\pause
%\begin{enumerate}
%\item the  expression contrasts negatively with the expected (polite) behavior for the context
%\pause
%\item this fact is generally well known among members of the linguistic community
%\end{enumerate}
%\pause
%\item The use of the expression then `signals' that the speaker's lack of respect for the recipients social standing.
%
%\end{itemize}
%
%\end{frame}
%
%\begin{frame}{The Contrastive Choice Account}
%
%\begin{itemize}
%\vspace{-.5cm}
%\item Racists tend to use racial slurs, rather than their neutral counterparts, when in the company of fellow racists.
%\pause
%\item Racial slurs are strongly associated with racist attitudes.
%\pause
%\item Since competent speakers are aware of this association (Condition 2), if a speaker uses a racial slur in a context in which the neutral counterpart is available and expected (condition 1), there is warrant for offense.
%\pause
%\item The speaker `signals' that she endorses the attitudes of racists. 
%\pause
%\item Offense can be undercut by the speaker's blameless ignorance of the slur's association or by forced-choice.
%\pause
%\item In such cases, offense may still be the \textit{rational} response.
%\end{itemize}
%
%\end{frame}
%
%\begin{frame}{The Contrastive Choice Account}
%
%\begin{block}{Explanatory Power}
%\begin{itemize}
%\item Explains derogatory autonomy,	attitude expression, and offensive variation (signaled attitudes differ in strength)	
%\pause
%\item Explains embedding failures: the speaker is still choosing the slur rather than its neutral counterpart.
%\pause
%\item Explains insulation: in cases of direct quotation or dictionary-type mentions, word-choice is forced.
%\pause
%\item Explains bystander complicity: use of slur generates expectations that participants share derogatory attitudes.
%
%\end{itemize}
%\end{block}
%
%
%\end{frame}










%\bibliography{aisb}

\end{document}
